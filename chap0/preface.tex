% 术语表

% 参考: https://zhuanlan.zhihu.com/p/273186198
%\label{chap:preface}
%\begin{table}[htbp]
%	\newcommand{\tabincell}[2]{\begin{tabular}{@{}#1@{}}#2\end{tabular}} %换行指令
%	\centering
%	\caption{名词列表 \label{tab:0_1}}
\renewcommand\arraystretch{1.0}	%设置表格内行间距
%\setlength{\tabcolsep}{4.5mm}{
\begin{longtable}{lll}
% https://blog.csdn.net/maths_girl/article/details/107030167
\caption{名词中英对照表 \label{tab:0_1}} \\
	\toprule 
 英文(缩略词)   && 中文 \\
 
 	\midrule
 	\makecell[l]{1-Methyl-4-phenyl-1,2,3,\\6-tetrahydropyridine (MPTP)}   && 焦磷酸甲基酯   \\
 	
 	\makecell[l]{2-amino-5-phosphonovaleric \\acid (APV)}    && \makecell[l]{2-氨基-\\5-膦酰基缬草酸}   \\
 	
 	\midrule
 	2-arachidonylglycerol (2-AG)   && 2-花生酰基甘油   \\
 	
 	\midrule
 	5$\alpha$-Reductase II deficiency   && 5$\alpha$-还原酶2缺乏症   \\
 
 	\midrule
 	5-hydroxyindoleacetic acid (5-HIAA)     && \href{https://baike.baidu.com/item/5-\%E7\%BE%9F%E5%9F%BA%E5%90%B2%E5%93%9A%E4%B9%99%E9%85%B8/16984024}{5-羟基吲哚乙酸}    \\
 	
 	\midrule
 	5-hydroxytryptophan (5-HT)     && \href{https://baike.baidu.com/item/5-\%E7%BE%9F%E5%9F%BA%E8%89%B2%E6%B0%A8%E9%85%B8/5687636}{5-羟色氨}    \\
 	
 	\midrule
 	5$\alpha$-Dihydrotestosterone (DHT)    && 5$\alpha$-二氢睾酮   \\
 	
 	\midrule
 	\makecell[l]{6-cyano-7-nitroquinoxaline-\\2,3-dione (CNQX)}    && \makecell[l]{6-氰基-7-硝基喹喔啉\\-2,3-二酮}   \\
 	
 	\midrule
 	6-n-propyl-2-thiouracil (PROP)   && 6-丙基-2-硫代尿嘧啶   \\
 	
 	\midrule
 	7q11.23    && \makecell[l]{第7号染色体\\1区1带\\2亚带3次亚带}   \\
 	
 	\midrule
 	\makecell[l]{8-hydroxy-diprolamino-tetraline\\ (8-OHDPAT)}    && 8-羟基-二丙醇氨基-四氢萘   \\
 	
 	% 阿尔茨海默病
 	\midrule
 	\makecell[l]{$\alpha$-amino-3-hydroxy-5-\\methyl-4-isoxazolepropionic acid (AMPA)}   && \makecell[l]{$\alpha$-氨基-3-羟基-5-\\甲基-4-异恶唑丙酸}   \\
 	
 	\midrule
 	$\alpha$-melanocyte-stimulating hormone ($\alpha$-MSH)    && $\alpha$-黑素细胞刺激素   \\
 	
 	\midrule
 	$\gamma$-aminobutyric acid (GABA)    && $\gamma$-氨基丁酸   \\
 	
 	\midrule
 	a disintegrin and metalloproteinase (ADAM)    && 一种去整合蛋白和金属蛋白酶   \\
 	
 	\midrule
 	A kinase attachment proteins     && A型激酶锚定蛋白   \\
 	
 	\midrule
 	Aaron Beck     && 亚伦$\cdot$贝克   \\
 	
 	\midrule
 	\makecell[l]{ATP-binding cassette transporter\\ (ABC transporter)}     && ATP结合盒式转运蛋白   \\
 	
 	\midrule
 	abdominals (ABD)     && 腹肌   \\
 
 	\midrule
 	Abducens nerve (abducens)     && 外旋神经   \\
 	
 	\midrule
 	abducens nucleus     && 外旋神经核   \\
 
 	\midrule
 	abduction     && 外转(角膜向外的眼球运动)   \\
 	
 	\midrule
 	abductor pollicis brevis (APB)     && 拇短展肌   \\
 	
 	\midrule
 	Abraham Lincoln     && 亚伯拉罕$\cdot$林肯   \\
 	
 	\midrule
 	Absence epilepsy     && 失神性癫痫   \\
 	
 	\midrule
 	absence seizure     && 失神性癫痫   \\
 	
 	% 从一个动作电位起始点开始的一段时间,在这段时间内另一个动作电位不能被再次触发
 	\midrule
 	absolute refractory period     && 绝对不应期   \\
 	
 	% XI
 	\midrule
 	Accessory nerve   &&  副神经   \\
 	
 	\midrule
 	Accessory olfactory bulb  (AOB) &&  副嗅球   \\
 	
 	\midrule
 	Accessory optic nuclei   && 视副核   \\
 	
 	% 通过改变晶状体的形状而使光线聚焦
 	\midrule
 	accommodation   && 眼调焦   \\
 
 	% 起神经递质作用的一种胺,存在于外周和中枢神经系统的许多突触中,包括神经肌接头之中
	\midrule
	acetylcholine (ACh)     && 乙酰胆碱   \\
	
	\midrule
	acetylcholinesterase (AChE)     && 乙酰胆碱酯酶   \\
	
	\midrule
	Acetylcholine receptors (AChR)    && 乙酰胆碱受体   \\
	
	% 一种存在于所有细胞中的细胞骨架蛋白,也是肌纤维中主要的细肌丝蛋白;它通过与肌球蛋白之间特殊外学作用而引起肌肉的收缩
	\midrule
	Actin, Actinin  && 肌动蛋白	   \\
	
	\midrule
	action potential  (AP)  && 动作电位   \\
	
	% 突触前膜的膜分化物,神经递质释放的位点
	\midrule
	active zone  && 活性带   \\
	
	\midrule
	activity-regulated cytoskeleton (ARC) && 活性调节的细胞骨架   \\
	
	\midrule
	Adaptor protein     && 衔接蛋白   \\
	
	\midrule
	Adenine (A)     && 腺嘌呤   \\
	
	\midrule
	adenosine triphosphate (ADP)     && 二磷酸腺苷   \\
	
	\midrule
	adenosine triphosphatase     && 腺苷三磷酸酶   \\
	
	\midrule
	adenosine triphosphatases (ATPase)   && 三磷酸腺苷酶   \\
	
	\midrule
	adenosine triphosphate (ATP)     && 三磷酸腺苷   \\
	
	\midrule
	adeno-associated virus (AAV)   && 腺相关病毒   \\
	
	\midrule
	Adenylyl cyclase     && 腺苷酸环化酶   \\
	
	\midrule
	adduction     && 内转   \\
	
	% 肾上腺的皮质部,在垂体促肾上腺皮质激素(pituitary adrenocorticortropic hormone)的刺激下释放皮质醇
	\midrule
	adrenal cortex     && 肾上腺皮质   \\
	
	% 肾上腺的中央部,受交感神经节前纤维的支配,释放肾上腺素(adrenaline或epinephrine)。
	\midrule
	adrenal medulla     && 肾上腺髓质   \\
	
	% 儿茶酚胺类神经递质,由去甲肾上腺素合成,也称epinephrine
	\midrule
	adrenaline     && 肾上腺素   \\
	
	% 当垂体前叶受到促肾上腺皮质激素释放激素的作用时释放的一种激素,可刺激肾上腺释放皮质醇
	\midrule
	adrenocorticotropic hormone (ACTH)     && \href{https://baike.baidu.com/item/\%E4%BF%83%E8%82%BE%E4%B8%8A%E8%85%BA%E7%9A%AE%E8%B4%A8%E6%BF%80%E7%B4%A0/2388734}{促肾上腺皮质激素}   \\
	
	\midrule
	Adrian Garcia-Sierra     &&  阿德里安$\cdot$·加西亚-塞拉  \\
	
	% 为了争夺配偶或保护后代,而不是为食物而发起的攻击,伴交感神经系统活性增强的现象,一般都会发出叫声,并且摆出威胁性或防御性的姿势
	\midrule
	affective aggression     &&  情感性攻击  \\
	
	% 一种以情绪障碍为特征的精神性情绪障碍,也称为心境障碍(mood disorder),例如重性抑郁症(major depression)和双相障碍(bipolar disorder)
	\midrule
	affective disorder     &&  情感障碍  \\
	
	% 研究心境(mood)和情绪(emotion)神经基础的科学
	\midrule
	affective neuroscience     &&  情感神经科学  \\
	
	\midrule
	afferent neuron     &&  传入神经元  \\
	
	% 动作电位的最后部分,即出现在细胞膜强烈去极化之后的超极化过程,也称为回射(undershoot)
	\midrule
	after-hyperpolarization     &&  后超极化  \\
	
	% 不能识别物体,但简单的感觉技巧可表现正常;通常由大脑后顶叶区的损伤所引起
	\midrule
	agnosia     &&  失认症  \\
	
	\midrule
	agonist–antagonist     &&  兴奋-拮抗  \\
	
	% 一种精神疾病,以患者身处难以逃离或者感到尴尬的场合时而出现严重的焦虑为特征
	\midrule
	agoraphobia     &&  广场恐怖症  \\
	
	\midrule
	agoutirelated peptide (AgRP)    &&  促食欲相关肽  \\
	
	\midrule
	agrin     &&  聚集蛋白  \\
	
	\midrule
	Alan Hodgkin    &&  艾伦$\cdot$霍奇金  \\
	
	% (胚胎)翼板
	\midrule
	Alar plate     &&  翼板  \\
	
	\midrule
	Albert Aguayo     &&  阿尔伯特$\cdot$阿瓜约  \\
	
	\midrule
	Albert Ellis     &&  阿尔伯特$\cdot$埃利斯  \\
	
	\midrule
	Alden Spencer     &&  奥尔登$\cdot$斯宾塞  \\
	
	\midrule
	Alexander Chesler     &&  亚历山大$\cdot$切斯勒  \\
	
	\midrule
	Alexander Luria     &&  亚历山大$\cdot$鲁利亚  \\
	
	\midrule
	Alfred Kohn     &&  阿尔弗雷德$\cdot$科恩  \\
	
	\midrule
	Alison Adcock     &&  艾丽森$\cdot$阿德科克  \\
	
	\midrule
	Allan Rechtschaffen     &&  艾伦$\cdot$雷克尚芬  \\
	
	\midrule
	alleles     &&  等位基因  \\
	
	\midrule
	all-or-none     &&  全有或全无  \\
	
	\midrule
	alien hand syndrome (AHS)     &&  异己手综合症  \\
	
	\midrule
	Alois Alzheimer     &&  阿尔茨海默  \\
	
	% 支配骨骼肌 梭外 纤维的神经元
	\midrule
	alpha motor neuro     &&  $\alpha$运动神经元  \\
	
	\midrule
	Altman     &&  奥特曼  \\
	
	\midrule
	Alzheimer disease (AD)     &&  阿尔茨海默病  \\
	
	\midrule
	Åke Vallbo     &&  亚克$\cdot$威尔波  \\
	
	% 眼视网膜内网状层的一种神经元,其神经突起侧向地投射
	\midrule
	amacrine cell     && 无长突细胞   \\
	
	\midrule
	Ameslan (ASL)    && 美国手语   \\
	
	% 蛋白质分子的化学构成单位,由一个核心碳原子、一个氨基基团、一个羧基基团和一个可变的R基团组成
	\midrule
	amino acid   && 氨基酸   \\
	
	\midrule
	amino terminus (H$_2$N)   && 氨基末端   \\
	
	\midrule
	Amiram Grinvald    && 阿米拉姆$\cdot$格林弗德   \\
	
	% 海马中的一个神经元薄层,发出轴突至穹隆
	\midrule
	Ammon's horn    && 阿蒙角   \\
	
	% 严重的记忆或学习能力丧失
	\midrule
	amnesia    && 遗忘症   \\
	
	\midrule
	amnesic patients (AMN)    && 失忆症患者   \\
	
	% 半规管的膨大部,内含有转导旋转刺激能量的毛细胞
	\midrule
	ampulla    && 壶腹   \\
	
	% 前颞叶皮层中的一个呈杏仁状的核团,被认为参与情绪以及某些类型的学习和记忆活动
	\midrule
	amygdala; amygdaloid (Am)    && 杏仁核   \\
	
	\midrule
	amyloid plaque     && 淀粉状蛋白斑   \\
	
	\midrule
	amyloid precursor protein (APP)     && 淀粉样前体蛋白   \\
	
	\midrule
	\makecell[l]{amyotrophic lateral sclerosis\\ (ALS), Lou Gehrig disease}    && \href{https://baike.baidu.com/item/\%E8\%82%8C%E8%90%8E%E7%BC%A9%E4%BE%A7%E7%B4%A2%E7%A1%AC%E5%8C%96/9336045}{肌萎缩侧索硬化}   \\
	
	% 从营养性前体物质合成生物有机分子,也称为anabolic metabolism
	\midrule
	anabolism     && 合成代谢   \\
	
	% 正常痛感觉的缺失
	\midrule
	analgesia     && 镇痛、痛觉缺失   \\
	
	\midrule
	anandamide     && 大麻素   \\
	
	\midrule
	Anatjari Tjampitjinpa     && 安纳塔里$\cdot$詹皮金帕   \\
	
	\midrule
	Anders Lundberg     && 安德斯$\cdot$伦德伯格   \\
	
	\midrule
	Andrew Huxley     && 安德鲁$\cdot$赫胥黎   \\
	
	\midrule
	Andrew Pruszynski     && 安德鲁$\cdot$普鲁斯钦斯基   \\
	
	\midrule
	Andrew Schally     && 安德鲁$\cdot$沙利   \\
	
	% 雄性类固醇激素的总称,其中以睾酮最为重要
	\midrule
	androgens     && 雄激素   \\
	
	\midrule
	androstadienone (AND)     && 雄二烯酮   \\
	
	\midrule
	Andy Young     && 安迪$\cdot$杨   \\
	
	\midrule
	Angela Friederici     && 安吉拉$\cdot$弗里德里希   \\
	
	\midrule
	Angelman Syndrome     && \href{https://baike.baidu.com/item/\%E5%A4%A9%E4%BD%BF%E7%BB%BC%E5%90%88%E5%BE%81/4662845}{天使综合症}   \\
	
	\midrule
	angiotensin (ANG)     && 血管紧张素   \\
	
	\midrule
	angle of the foot in space     && 脚在空间中的角度   \\
	
	\midrule
	annulus of Zinn     && 总腱环   \\
	
	\midrule
	anterior cingulate cortex (ACC)     && 前扣带皮层   \\
	
	\midrule
	anterior commissure (AC)     && 前连合   \\
	
	\midrule
	anterior insula (AI)     && 前脑岛   \\
	
	\midrule
	anterior intraparietal area (AIP)     && 前顶叶   \\
	
	\midrule
	anterior lateral (AL)     && 前外侧   \\
	
	\midrule
	anterior thalamus (antTHAL)     && 前丘脑   \\
	
	\midrule
	antisense oligonucleotides (ASO, ASOs)     && 反义寡核苷酸   \\
	
	\midrule
	anterior burster  (AB)   && 	前部滑囊器   \\
	
	\midrule
	Anterior commissure     && 	前连合   \\
	
	\midrule
	Anterior cranial fossa     && 	颅前窝   \\
	
	\midrule
	anterior medial     && 	前内侧   \\
	
	\midrule
	anterior tibial muscle     && 	胫骨前肌   \\
	
	\midrule
	anterograde amnesia     && 	顺行性遗忘症   \\
	
	\midrule
	Anthony Movshon     && 	安东尼$\cdot$莫夫松   \\
	
	\midrule
	antigravity muscles     && 	抗引力肌   \\
	
	\midrule
	Anxiety Sensitivity Index     && 	焦虑度   \\
	
	\midrule
	Aplysia     && 海兔   \\
	
	\midrule
	Apperceptive agnosia     && 统觉性失认症   \\
	
	\midrule
	apolipoprotein E (APOE)     && \href{https://baike.baidu.com/item/\%E8%BD%BD%E8%84%82%E8%9B%8B%E7%99%BDE/4226374}{载脂蛋白E}   \\
	
	\midrule
	apoptosis activating factor-1 (Apaf-1)     && 凋亡蛋白酶激活因子1   \\
	
	\midrule
	arachidonic acid     && 花生四烯酸   \\
	
	\midrule
	Area Under Curve (AUC)     && 曲线下面积   \\
	
	\midrule
	arcuate fasciculus     && 弓状束   \\
	
	\midrule
	Arcuate sulcus     && 弓形沟   \\
	
	\midrule
	Ardem Patapoutian     && \href{https://baike.baidu.com/item/%E9%9B%85%E9%A1%BF%C2%B7%E5%B8%95%E5%A1%94%E6%99%AE%E8%92%82%E5%AE%89/58754597}{雅顿$\cdot$帕塔普蒂安}   \\
	
	\midrule
	aristaless related homeobox (ARX)    && Aristaless相关同源异型盒基因   \\
	
	\midrule
	arrestin     && 抑制蛋白   \\
	
	\midrule
	Arthur Ewins     && 亚瑟$\cdot$埃文斯   \\
	
	\midrule
	Arthur Karlin     && 亚瑟$\cdot$卡林   \\
	
	\midrule
	Arvid Carlsson     && 阿尔维德$\cdot$卡尔松   \\
	
	\midrule
	AsbØrn FØlling     && 阿斯比约恩$\cdot$佛林   \\
	
	\midrule
	Ascending tract of Deiters     && 戴特氏上行束   \\
	
	\midrule
	aspartic acid (Asp)     && 天冬氨酸   \\
	
	\midrule
	Asperger syndrome     && 阿斯伯格综合症   \\
	
	\midrule
	Associative learning     && 联想学习   \\
	
	\midrule
	Ataxin-1 (ATXN1)     && 共济失调蛋白-1   \\
	
	\midrule
	Ativan     && 劳拉西泮   \\
	
	\midrule
	\makecell[l]{Attention deficit hyperactivity\\ disorder (ADHD)}     && 注意缺陷多动障碍   \\
	
	\midrule
	autism spectrum disorder (ASD)     && \href{https://baike.baidu.com/item/\%E8%87%AA%E9%97%AD%E7%97%87%E8%B0%B1%E7%B3%BB%E9%9A%9C%E7%A2%8D/1704369}{孤独症谱系障碍}   \\
	
	\midrule
	autistic savant     && 孤独症学者   \\
	
	\midrule
	autosomal dominant  (AD)   && 常染色体显性   \\
	
	\midrule
	\makecell[l]{autosomal dominant nocturnal \\ frontal lobe epilepsy (ADNFLE)}    && \makecell[l]{常染色体显性遗传\\夜间额叶癫痫}   \\
	
	\midrule
	autosomal recessive  (AR)   && 常染色体隐性   \\
	
	\midrule
	Axonal transport     && 轴突运输   \\
	
	\midrule
	Balint syndrome   && 巴林特综合症  \\
	
	\midrule
	Barnes maze   && 巴恩斯迷宫  \\
	
	\midrule
	Barrel cortex   && 桶状皮层  \\
	
	% 巴灵顿氏核
	\midrule
	Barrington’s nucleus   && 丘脑前核群   \\
	
	\midrule
	basal ganglia   && 基底核  \\
	
	\midrule
	basal nucleus of Meynert   && 迈纳特基底核  \\
	
	\midrule
	Basal plate   && 基板  \\
	
	\midrule
	Basal temporal   && 基底颞叶  \\
	
	\midrule
	Base Pair (bp)  && 碱基对  \\
	
	\midrule
	basic helix-loop-helix (bHLH)  && 碱性螺旋环螺旋  \\
	
	\midrule
	Basis pedunculi   && 基脚  \\
	
	\midrule
	Basket cell   && 篮状细胞  \\
	
	\midrule
	Beck Anxiety Inventory   && 贝克焦虑问卷  \\
	
	\midrule
	Becker muscular dystrophy   && 贝克肌营养不良  \\
	
	\midrule
	bed nucleus of the stria terminalis (BNST)  && 终纹床核  \\
	
	% 习惯上将较易以量的变化来表达身体的发育称为身体的生长(physical growth),而将与质相联系的行为型及机能的系列变化多称为行为发育(behavior development)。
	\midrule
	behavior development && 行为发育  \\
	
	\midrule
	Ben Barres   && 本$\cdot$巴瑞斯  \\
	
	\midrule
	Bengt Falck   && 本特$\cdot$法尔克  \\
	
	\midrule
	Benjamin Brodie   && 本杰明$\cdot$布罗迪  \\
	
	\midrule
	Benjamin Libet   && 本杰明$\cdot$李贝特  \\
	
	\midrule
	Benoni Edin   && 贝诺尼$\cdot$爱丁  \\
	
	\midrule
	Bernard Katz   && 伯纳德$\cdot$卡茨  \\
	
	\midrule
	Bert Sakmann   && 伯特$\cdot$萨克曼  \\
	
	\midrule
	Bertil Hille   && 贝蒂尔$\cdot$希勒  \\
 
	\midrule
	Best Frequency (BF)     && 最佳频率   \\
	
	\midrule
	Bethlem myopathy     && 贝斯勒姆肌病   \\
	
	\midrule
	biceps brachii     && 	肱二头肌   \\
	
	\midrule
	bimodal neuron     && 	双模态神经元   \\
	
	\midrule
	Bithorax complex     && 	双胸复合体   \\
	
	% 昼夜节律转录因子
	\midrule
	BMAL1     && 	脑和肌肉芳烃受体核转位蛋白1   \\
	
	\midrule
	\makecell[l]{body angle relative to the foot (BF)}     && 	身体相对于脚的角度   \\
	
	\midrule
	\makecell[l]{body tilt with respect to\\ earth vertical or \\body-in-space (BS)}  && 	\makecell[l]{身体相对于地球\\垂直倾斜或身体\\在空间中倾斜}   \\
	
	\midrule
	bone morphogenetic protein  (BMP)   && 	骨形态发生蛋白   \\
	
	\midrule
	\makecell[l]{bone morphogenetic protein \\ receptors  (BMPR)}   && 	骨形态发生蛋白受体   \\
 
	\midrule
	Boold Oxygen-Level Dependent (BOLD)     && 血氧水平依赖   \\
	
	\midrule
	Border cell     && 边界细胞   \\
	
	\midrule
	brachioradialis (BR)    && 肱桡肌   \\
	
	\midrule
	bradykinin (BK)     && 缓激肽   \\
	
	\midrule
	Brahms     && \href{https://baike.baidu.com/item/%E7%BA%A6%E7%BF%B0%E5%86%85%E6%96%AF%C2%B7%E5%8B%83%E6%8B%89%E5%A7%86%E6%96%AF/581682?fromtitle=%E5%8B%83%E6%8B%89%E5%A7%86%E6%96%AF&fromid=345657}{勃拉姆斯}   \\
	
	\midrule
	brain derived neurotrophic factor (BDNF)     && 脑源性神经营养因子   \\
	
	\midrule
	BRAIN Initiative     && 脑计划   \\
	
	\midrule
	brain stem (BS)     && 脑干   \\
	
	\midrule
	Brenda Milner     && 布伦达$\cdot$米尔纳   \\
	
	\midrule
	British Sign Language (BSL)     && 英国手语   \\
	
	\midrule
	Brodmann area (Cg25)   && 布罗德曼 25 区  \\
	
	\midrule
	Brody myopathy   && 布罗迪肌病  \\
	
	\midrule
	Bror Alstermark   && 布鲁尔$\cdot$阿尔斯特马克  \\
	
	\midrule
	Bruckner   && \href{https://baike.baidu.com/item/%E5%AE%89%E4%B8%9C%C2%B7%E5%B8%83%E9%B2%81%E5%85%8B%E7%BA%B3/584381?fr=ge_ala}{布鲁克纳}  \\
	
	\midrule
	Bulbocavernosus muscle  && 球海绵体肌  \\
	
	\midrule
	Bungarus multicinctus  && 银环蛇  \\
	
	\midrule
	B. F. Skinner  && 斯金纳  \\
	
	
	\midrule
	Caenorhabditis elegans (C. elegans)  && \href{https://baike.baidu.com/item/\%E7%A7%80%E4%B8%BD%E9%9A%90%E6%9D%86%E7%BA%BF%E8%99%AB/154672}{秀丽隐杆线虫}  \\
	
	\midrule
	Cajal-Retzius cells  && 卡-雷氏细胞  \\
	
	\midrule
	calcitonin gene–related peptide (CGRP)   && 降钙素基因相关肽  \\
	
	\midrule
	\makecell[l]{Calcium-calmodulin (CaM)-dependent \\protein kinase II (CaMKII)}   && \makecell[l]{钙/钙调蛋白\\依赖性蛋白激酶 2}  \\
	
	\midrule
	calmodulin (CaM)   && 钙调蛋白  \\
	
	\midrule
	Calyx of Held   && 花萼突触  \\
	
	\midrule
	Camillo Golgi   && 卡米洛$\cdot$高尔基  \\
	
	\midrule
	cAMP response element (CRE)   && 环磷酸腺苷应答元件  \\
	
	\midrule
	\makecell[l]{cAMP response element binding\\ protein (CREB)}  && \makecell[l]{环磷酸腺苷应答元件\\结合蛋白}  \\
	
	\midrule
	Cannon-Bard   && 坎农-巴德  \\
	
	\midrule
	Canonical babbling   && 咿呀学语  \\
	
	\midrule
	canonical splice site mutation   && 经典剪切位点突变  \\
	
	\midrule
	Capgras Sydrome   && 替身综合症  \\
	
	\midrule
	Capicua (CIC)   && Capicua转录阻抑蛋白  \\
	
	\midrule
	Carandini   && 卡兰迪尼  \\
	
	\midrule
	carboxy terminus (COOH)   && 羧基末端  \\
	
	\midrule
	Carl Olson   && 卡尔$\cdot$奥尔森  \\
	
	\midrule
	Carl Wernicke   && 卡尔$\cdot$韦尼克  \\
	
	\midrule
	casein-1 kinase epsilon or delta   && 酪蛋白-1激酶$\epsilon$或$\delta$  \\
	
	\midrule
	Caspase   && 半胱天冬酶  \\
	
	\midrule
	catalytic subunit (C)   && 催化亚基  \\
	
	\midrule
	catechol-O-methyltransferase (COMT)  && 儿茶酚氧位甲基转移酶  \\
	
	\midrule
	categorical perception  && \href{https://baike.baidu.com/item/%E5%88%86%E7%B1%BB%E7%9F%A5%E8%A7%89/62624331}{分类知觉}  \\
	
	\midrule
	catenin  && 连环蛋白  \\
	
	\midrule
	Caveolin   && 凹陷蛋白  \\
	
	\midrule
	Cavernous sinus   && 海绵窦  \\
	
	\midrule
	CB1 cannabinoid receptors   && 大麻素受体  \\
	
	\midrule
	cell death (ced)  && 细胞死亡  \\
	
	\midrule
	center of mass (CoM)   && 质心  \\
	
	\midrule
	center of pressure (CoP)   && 压力中心  \\
	
	\midrule
	Central core disease  && 中央轴空病  \\
	
	\midrule
	central gray region (CG)  && 中央灰质区  \\
	
	\midrule
	central nervous system (CNS)  && 中枢神经系统  \\
	
	\midrule
	Central neuron   && 中枢神经元  \\
	
	\midrule
	central pattern generators (CPGs)   && 中枢模式发生器  \\
	
	\midrule
	Cephalic flexure   && 头曲  \\
	
	\midrule
	Cerebellar flocculus   && 小脑小叶  \\
	
	\midrule
	Cerebellar peduncles   && 小脑脚  \\
	
	\midrule
	Cerebellopontine angle (CPA)   && 桥小脑角区  \\
	
	\midrule
	cerebral amyloid angiopathy (CAA)   && 脑淀粉样血管病变  \\
	
	\midrule
	cerebrospinal fluid (CSF)   && 脑脊液  \\
	
	\midrule
	Cervical flexure   && 颈曲  \\
	
	\midrule
	Cervical ventral roots   && 颈腹根  \\
	
	\midrule
	cGMP-dependent protein kinase (PKG)   && 环磷酸腺苷依赖蛋白激酶  \\
	
	\midrule
	chandelier cell   && 吊灯细胞  \\
	
	\midrule
	change blindness   && \href{https://baike.baidu.com/item/%E5%8F%98%E5%8C%96%E7%9B%B2%E8%A7%86/10083810?fr=ge_ala}{变化盲}  \\
	
	\midrule
	channelrhodopsin-2   && 光敏感通道蛋白  \\
	
	\midrule
	Chaperone proteins   && 伴侣蛋白  \\
	
	\midrule
	\makecell[l]{peronial myoatrophy \\ (Charcot-Marie-Tooth, CMT)}   && 遗传性神经性肌萎缩  \\
	
	\midrule
	Charcot-Marie-Tooth disease   && 腓骨肌萎缩症  \\
	
	\midrule
	Charles Bell   && \href{https://baike.baidu.com/item/%E6%9F%A5%E5%B0%94%E6%96%AF%C2%B7%E8%B4%9D%E5%B0%94/3328954}{查尔斯$\cdot$贝尔}  \\
	
	\midrule
	Charles Bonnet   && 查尔斯$\cdot$庞奈  \\
	
	\midrule
	Charles Darwin   && 查尔斯$\cdot$达尔文  \\
	
	\midrule
	Charles F. Stevens   && 查尔斯$\cdot$史蒂芬斯  \\
	
	\midrule
	Charles Gilbert   && 查尔斯$\cdot$吉尔伯特  \\
	
	\midrule
	Charles Gross   && 查尔斯$\cdot$格罗斯  \\
	
	\midrule
	Charles Sherrington   && 查尔斯$\cdot$谢林顿  \\
	
	\midrule
	Cheyne-Stokes respiration   && 潮式呼吸  \\
	
	\midrule
	Chiara Cirelli   && 基娅拉$\cdot$奇雷利  \\
	
	\midrule
	Chippendale side chair   && 奇彭代尔式侧椅  \\
	
	\midrule
	Cholecystokinin (CCK)   && 胆囊收缩素  \\
	
	\midrule
	choline (Ch)   && 胆碱  \\
	
	\midrule
	choline acetyltransferase (ChAT)   && 胆碱乙酰转移酶  \\
	
	\midrule
	choline transporter (CHT)   && 胆碱转运蛋白  \\
	
	\midrule
	\makecell[l]{Chondroitin sulphate proteoglycans \\(CSPG)} && 硫酸软骨素蛋白多糖  \\
	
	\midrule
	Chorda tympani nerve   && 鼓索神经  \\
	
	\midrule
	Chorea-acanthocytosis   && 舞蹈病  \\
	
	\midrule
	choroid plexus (CP)   && 脉络丛  \\
	
	\midrule
	Christopher Koch   && 克里斯托弗$\cdot$科赫  \\
	
	\midrule
	Christopher Koch   && 克里斯托弗$\cdot$米勒  \\
	
	\midrule
	chronic traumatic encephalopathy (CTE)   && 慢性创伤性脑病  \\
	
	\midrule
	ciliary neurotrophic factor (CNTF)  && 睫状神经营养因子  \\
	
	\midrule
	cingulate motor areas (CMA)   && 扣带运动区  \\
	
	\midrule
	Claude Bernard   && 克劳德$\cdot$伯纳德  \\
	
	\midrule
	Clay Armstrong   && 克莱$\cdot$阿姆斯特朗  \\
	
	\midrule
	Climbing fiber (CF)  && 攀缘纤维  \\
	
	\midrule
	Clinton Woolsey  && 克林顿$\cdot$伍尔西  \\
	
	\midrule
	Clive Waring   && 克里夫$\cdot$韦林  \\
	
	\midrule
	clock gene   && 时钟基因  \\
	
	\midrule
	CLOCK protein (CLOCK)  && 时钟蛋白  \\
	
	\midrule
	\makecell[l]{Clustered Regularly Interspaced\\ Short Palindromic Repeats (CRISPR)}  && \makecell[l]{规律成簇的\\间隔短回文重复}  \\
	
	\midrule
	cm Hg  && 厘米汞柱  \\
	
	\midrule
	Coated pit  && \href{https://baike.baidu.com/item/%E5%8C%85%E8%A2%AB%E5%B0%8F%E7%AA%9D/53651932?fr=ge_ala}{包被小窝}  \\
	
	\midrule
	\makecell[l]{Cocaine- and amphetamineregulated\\ transcript (CART)}   && 可卡因和苯丙胺调节转录物  \\
	
	\midrule
	Cochlear Nucleus(CN)   && 耳蜗核  \\
	
	\midrule
	Commissural fibers   && 连合纤维  \\
	
	\midrule
	commissural interneurons (CIN)   && 连合中间神经元  \\
	
	\midrule
	Complement factor 3b (C3b)  && 补体因子3b  \\
	
	\midrule
	Complement factor 4 (C4)  && 补体因子4  \\
	
	\midrule
	complementary DNA (cDNA)   && 互补脱氧核糖核酸  \\
	
	\midrule
	\makecell[l]{Complete androgen insensitivity\\ syndrome (CAIS)}  && 完全型雄激素不敏感综合症  \\
	
	\midrule
	compound muscle action potential (CMAP) && 复合肌肉动作电位  \\
	
	\midrule
	computed tomographic (CT)   && 计算机断层扫描  \\
	
	\midrule
	conditioned stimulus (CS)     &&  条件刺激  \\
	
	\midrule
	Congenital myopathy     &&  先天性肌病  \\
	
	\midrule
	connectional specificity     &&  连接特异性  \\
	
	\midrule
	Constantin von Economo    &&  康斯坦丁$\cdot$冯$\cdot$艾克诺  \\
	
	\midrule
	Constant-Frequency (CF)     &&  恒频  \\
	
	\midrule
	cone cell      && 视锥细胞  \\
	
	\midrule
	Congenital adrenal hyperplasia (CAH)  && 先天性肾上腺皮质增生症  \\
	
	\midrule
	Congenital muscular dystrophy  && 先天性肌营养不良  \\
	
	\midrule
	connexin-32 (Cx32) && 连接蛋白32  \\
	
	\midrule
	Contactin && 接触蛋白  \\
	
	\midrule
	\makecell[l]{continuous positive airway \\ pressure (CPAP)}     && 持续气道正压  \\
	
	\midrule
	copy number variation (CNV)      && \href{https://baike.baidu.com/item/\%E6%8B%B7%E8%B4%9D%E6%95%B0%E5%8F%98%E5%BC%82}{拷贝数变异}  \\
	
	\midrule
	cornu Ammonis (CA)    &&  阿蒙角  \\
	
	\midrule
	corollary discharge     &&  \href{https://baike.baidu.com/item/%E4%BC%B4%E9%9A%8F%E5%8F%91%E9%80%81}{伴随发送}  \\
	
	\midrule
	coronal plane     &&  冠状面  \\
	
	\midrule
	corpus callosum     &&  胼胝体  \\
	
	\midrule
	cortical; cortex     &&  皮层  \\
	
	\midrule
	cortical plate (CP)     &&  皮层板  \\
	
	\midrule
	corticomotoneuronal (CM)     &&  皮层运动神经  \\
	
	\midrule
	corticospinal tract  (CST)   &&  皮层脊髓束  \\
	
	\midrule
	\makecell[l]{corticotropin-like \\intermediate-lobe peptide (CLIP)}  &&  促皮质激素样中间肽  \\
	
	\midrule
	Corticotropin-releasing factor (CRF)  &&  促皮质素释放因子  \\
	
	\midrule
	corticotropin-releasing hormone (CRH)    &&  \href{https://baike.baidu.com/item/\%E4%BF%83%E8%82%BE%E4%B8%8A%E8%85%BA%E7%9A%AE%E8%B4%A8%E6%BF%80%E7%B4%A0%E9%87%8A%E6%94%BE%E6%BF%80%E7%B4%A0/3760624}{促肾上腺皮质激素释放激素}  \\
	
	\midrule
	CO$_2$ partial pressure (PCO$_2$)  &&  二氧化碳分压  \\
	
	\midrule
	CREB binding protein  (CBP)   &&  环磷酸腺苷应答元件结合蛋白  \\
	
	\midrule
	Craig Barclay   &&  克雷格$\cdot$巴克莱  \\
	
	\midrule
	creatine phosphokinase (CPK)  &&  肌酸磷酸激酶  \\
	
	\midrule
	cryptochrome gene (cry)   &&  隐花色素基因  \\
	
	\midrule
	Crystallin   &&  晶体蛋白  \\
	
	\midrule
	Cribiform plate     &&  筛板  \\
	
	\midrule
	cue's location (PD)     &&  提示方向  \\
	
	\midrule
	cuneiform nucleus (CNF)     &&  楔形核  \\
	
	\midrule
	Cupula     &&  壶腹帽  \\
	
	\midrule
	current receptive field (CRF)     &&  当前感受野  \\
	
	\midrule
	Cuticular plate     &&  膜状板  \\
	
	\midrule
	cyclic adenosine monophosphate (cAMP)     &&  环磷酸腺苷  \\
	
	\midrule
	\makecell[l]{cyclic guanosine \\ 3′,5′-monophosphate (cGMP)}     &&  环鸟苷-3,5-单磷酸盐  \\
	
	\midrule
	Cyclic nucleotide–gated channel    &&  环核苷酸门控离子通道  \\
	
	\midrule
	cycloheximide (CYX)    &&  \href{https://baike.baidu.com/item/%E7%8E%AF%E5%B7%B1%E9%85%B0%E4%BA%9A%E8%83%BA/7239227?fr=ge_ala}{环己酰亚胺}  \\
	
	\midrule
	cyclooxygenase  (COX)  &&  \href{https://baike.baidu.com/item/%E7%8E%AF%E6%B0%A7%E5%8C%96%E9%85%B6/4753442}{环氧化酶}  \\
	
	\midrule
	cys-loop receptor    &&  半胱氨酸-环受体  \\
	
	\midrule
	Cys-S-S-Cys    &&  二硫键  \\
	
	\midrule
	cytomegalovirus    &&  巨细胞病毒  \\
	
	\midrule
	\makecell[l]{cytoplasmic polyadenylation element \\ binding protein  (CPEB)}   &&  胞质聚腺苷酸化元件结合蛋白  \\
	
	\midrule
	cytosine (C)     &&  胞嘧啶  \\
	
	\midrule
	cytosine-adenine-guanine (CAG)     &&  胞嘧啶-腺嘌呤-鸟嘌呤  \\
	
	\midrule
	C. Miller Fisher     &&  米勒$\cdot$费雪  \\
	
	\midrule
	c-Jun N-terminal kinase (JNK)    &&  c-Jun氨基末端激酶  \\
	
	\midrule
	Dahlstrom   &&  达尔斯特伦  \\
	
	% 分子量
	\midrule
	Dalton (Da)   &&  道尔顿  \\
	
	\midrule
	Daniel L. Yamins     &&  丹尼尔$\cdot$亚明斯  \\
	
	\midrule
	Daniel Wolpert     &&  丹尼尔$\cdot$沃普特  \\
	
	\midrule
	Daniela Perani     &&  丹妮拉$\cdot$佩拉尼  \\
	
	\midrule
	Danielle Bassett     &&  丹妮尔$\cdot$巴西特  \\
	
	\midrule
	dart and dome     &&  圆顶尖角  \\
	
	\midrule
	David Freedman     &&  大卫$\cdot$弗雷德曼  \\
	
	\midrule
	David Ferrier     &&  大卫$\cdot$费里尔  \\
	
	\midrule
	David Hubel     &&  大卫$\cdot$休伯尔  \\
	
	\midrule
	David Hume     &&  大卫$\cdot$休谟  \\
	
	\midrule
	David Ginty     &&  大卫$\cdot$金蒂  \\
	
	\midrule
	David Julius     &&  大卫$\cdot$朱利斯  \\
	
	\midrule
	David Lloyd     &&  大卫$\cdot$劳埃德  \\
	
	\midrule
	David Marr     &&  大卫$\cdot$马尔  \\
	
	\midrule
	David McCormick     &&  大卫$\cdot$麦考密克  \\
	
	\midrule
	David Milner     &&  大卫$\cdot$米尔纳  \\
	
	\midrule
	David Poeppel     &&  大卫$\cdot$波佩尔  \\
	
	\midrule
	David Potter     &&  大卫$\cdot$波特  \\
	
	\midrule
	David Prince     &&  大卫$\cdot$王子  \\
	
	\midrule
	David Rosenthal     &&  大卫$\cdot$罗森萨尔  \\
	
	\midrule
	de novo mutation     &&  新生突变  \\
	
	\midrule
	Deborah Mills     &&  黛博拉$\cdot$米尔斯  \\
	
	\midrule
	Deep brain stimulation (DBS)     &&  深部脑刺激  \\
	
	\midrule
	Deep cerebellar nucleus   &&  小脑深部核团  \\
	
	\midrule
	Deiters's cell   &&  戴特氏细胞  \\
	
	\midrule
	Deiters' nucleus   &&  戴特氏核  \\
	
	\midrule
	Dejerine-Roussy syndrome   &&  丘脑综合征;德热里纳-鲁西综合症  \\
	
	\midrule
	\makecell[l]{Dejerine-Sottas infantile neuropathy \\ (Dejerine-Sottas Syndrome)}  &&  \makecell{肥大性神经炎,\\德热里纳-索塔斯婴儿神经病}  \\
	
	\midrule
	Del Castillo   &&  德尔$\cdot$卡斯蒂略  \\
	
	\midrule
	delay eyeblink conditioning   &&  延迟性眨眼条件反射  \\
	
	\midrule
	delayed-match-to-sample  (DMS)   &&  延迟样本匹配  \\
	
	\midrule
	deleted in colon cancer (DCC)     &&  结直肠癌缺失  \\
	
	\midrule
	denatonium (DEN)   && 苯甲地那铵 \\
	
	\midrule
	dentate gyrus (DG)     && 齿状回 \\
	
	\midrule
	Dentate region     && 齿状区 \\
	
	\midrule
	\makecell[l]{dentatorubro-pallidoluysian\\ atrophy (DRPLA)}     &&  \makecell[l]{齿状核红核苍白球\\路易体萎缩症}  \\
	
	\midrule
	DeoxyriboNucleic Acid (DNA)     &&  脱氧核糖核酸  \\
	
	\midrule
	Depakote     &&  丙戊酸钠  \\
	
	\midrule
	depression, infraduction     &&  下转(角膜向下的眼球运动)  \\
	
	\midrule
	\makecell[l]{designer receptors exclusively\\ activated by designer drugs (DREADDs)}    &&  设计药物激活的设计受体  \\
	
	\midrule
	desmin   &&  肌间线蛋白  \\
	
	\midrule
	Desmin storage myopathy   &&  肌间线蛋白储存性肌病  \\
	
	\midrule
	diacylglycerol  (DAG)   &&  甘油二酯  \\
	
	\midrule
	diacylglycerol lipase  (DAGL)   &&  二酰基甘油脂酶  \\
	
	\midrule
	\makecell[l]{Diagnostic and Statistical Manual of \\Mental Disorders}     &&  《精神疾病诊断和统计手册》  \\
	
	\midrule
	\makecell[l]{Diagnostic and Statistical Manual of \\Mental Disorders, Fifth Edition (DSM-5)}     &&  \makecell[l]{《精神疾病诊断和统计手册\\(第五版)》}  \\
	
	\midrule
	Diana Bautista    &&  戴安娜$\cdot$保蒂斯塔  \\
	
	\midrule
	diffuse bipolar cell (DB)    &&  弥散双极细胞  \\
	
	\midrule
	diffusion tensor imaging (DTI)    &&  弥散张量成像  \\
	
	\midrule
	DiGeorge syndrome    &&  迪格奥尔格综合症  \\
	
	\midrule
	Dilantin    &&  苯妥英钠  \\
	
	\midrule
	Distal myofibrillar myopathy    &&  远端肌纤维性肌病  \\
	
	\midrule
	DMD    &&  进行性肌营养不良症  \\
	
	\midrule
	Dominic ffytche    &&  多米尼克$\cdot$费切  \\
	
	\midrule
	Donald Hebb    &&  唐纳德$\cdot$赫布  \\
	
	\midrule
	Donald Humphrey    &&  唐纳德$\cdot$汉弗莱  \\
	
	\midrule
	Donald Rudin    &&  唐纳德$\cdot$鲁丁  \\
	
	\midrule
	dopamine transporter (DAT)     &&  多巴胺转运蛋白  \\
	
	\midrule
	Doppler-shifted constant-frequency (DSCF)     &&  多普勒频移恒频  \\
	
	\midrule
	Doris Tsao     &&  曹颖  \\
	
	\midrule
	dorsal anterior cingulate cortex (dACC)     &&  背前扣带皮层  \\
	
	\midrule
	dorsal caudate nucleus (dCN)     &&  背侧尾核  \\
	
	\midrule
	dorsal premotor cortex (PMd)     &&  背侧前运动皮层  \\
	
	\midrule
	dorsal raphe (DR)     &&  中缝背核  \\
	
	\midrule
	dorsal root ganglion (DRG)     &&  背根神经节  \\
	
	\midrule
	dorsal spinocerebellar tract (DSCT)     &&  脊髓小脑背侧束  \\
	
	\midrule
	dorsolateral prefrontal cortex (DLPFC, F9)     &&  背外侧前额叶皮层  \\
	
	\midrule
	dorsomedial prefrontal cortex (DMPFC)     &&  背内侧前额叶皮层  \\
	
	\midrule
	\makecell[l]{dorsomedial nucleus of \\the hypothalamus (DMH)}    &&  下丘脑背内侧核  \\
	
	\midrule
	dorsomedial thalamus (dmTHAL)     &&  背内侧丘脑  \\
	
	\midrule
	doublecortin     &&  双皮质素  \\
	
	\midrule
	Dorsum sellae     &&  鞍背  \\
	
	\midrule
	Douglas Coleman     &&  道格拉斯$\cdot$科尔曼  \\
	
	\midrule
	Down syndrome     &&  唐氏综合症  \\
	
	\midrule
	Doxycycline     &&  多西环素  \\
	
	\midrule
	Dravet syndrome     &&  婴儿严重肌阵挛性癫痫  \\
	
	\midrule
	Drosophila melanogaster     &&  黑腹果蝇  \\
	
	\midrule
	ductus reuniens     &&  黑腹果蝇  \\
	
	\midrule
	dynactin     &&  动力肌动蛋白  \\
	
	\midrule
	Dynorphin     &&  强啡肽  \\
	
	\midrule
	Duchenne     &&  杜氏营养不良症  \\
	
	\midrule
	Dura     &&  硬脑膜  \\
	
	\midrule
	dystroglycan     &&  肌营养不良蛋白聚糖  \\
	
	\midrule
	Earl Miller     &&  厄尔$\cdot$米勒  \\
	
	\midrule
	echo planar imaging (EPI)     &&  平面回波成像  \\
	
	\midrule
	electrotonic transmission     &&  电紧张传输   \\
	
	\midrule
	Ed Evarts     &&  埃德$\cdot$埃瓦茨  \\
	
	\midrule
	Edgar Adrian     &&  \href{https://baike.baidu.com/item/%E5%9F%83%E5%BE%B7%E5%8A%A0%C2%B7%E9%98%BF%E5%BE%B7%E9%87%8C%E5%AE%89/7722373}{埃德加$\cdot$阿德里安}  \\
	
	\midrule
	Edinger-Westphal nucleus     &&  动眼神经副核  \\
	
	% 一种抗胆碱酯酶药
	\midrule
	edrophonium     &&  腾喜龙  \\
	
	\midrule
	Eduard Hitzig     &&  爱德华$\cdot$希茨格  \\
	
	\midrule
	Edward Albert Schaefer     &&  爱德华$\cdot$艾伯特$\cdot$谢弗  \\
	
	\midrule
	Edward Evarts     &&  爱德华$\cdot$埃瓦茨  \\
	
	\midrule
	Edwin Furshpan     &&  爱德华$\cdot$弗斯潘  \\
	
	\midrule
	Edward Krebs     &&  爱德华$\cdot$克雷布  \\
	
	\midrule
	Edward Lee Thorndike     &&  爱德华$\cdot$李$\cdot$桑代克  \\
	
	\midrule
	Edward Tolman     &&  爱德华$\cdot$托尔曼  \\
	
	\midrule
	Edwin Landott     &&  爱德华$\cdot$兰德  \\
	
	\midrule
	Edvard Moser     &&  爱德华$\cdot$莫泽  \\
	
	\midrule
	Edward Purcell     &&  爱德华$\cdot$珀塞尔  \\
	
	\midrule
	efference copy     &&  传出副本  \\
	
	\midrule
	egg-laying hormone (ELH)    &&  产卵激素  \\
	
	\midrule
	elastin  (ELN)   &&  弹性蛋白  \\
	
	\midrule
	Eleanor Roosevelt     &&  \href{https://baike.baidu.com/item/%E5%AE%89%E5%A8%9C%C2%B7%E5%9F%83%E8%8E%89%E8%AF%BA%C2%B7%E7%BD%97%E6%96%AF%E7%A6%8F/243493}{埃莉诺$\cdot$罗斯福}  \\
	
	\midrule
	electroconvulsive therapy (ECT)     &&  电痉挛疗法  \\
	
	\midrule
	electrocorticography (ECoG)     &&  脑皮层电图  \\
	
	\midrule
	electroencephalogram (EEG)   &&  脑电图  \\
	
	\midrule
	electromyogram (EMG)     &&  肌电图  \\
	
	\midrule
	electromotive force (EMF)     &&  电动势  \\
	
	\midrule
	electrotonic potential     &&  电紧张电位  \\
	
	\midrule
	electro-oculogram (EOG)     &&  眼电图  \\
	
	\midrule
	elevation, supraduction     &&  上转(角膜向上的眼球运动)  \\
	
	\midrule
	Ellen Lumpkin      &&  艾伦$\cdot$伦普金  \\
	
	\midrule
	Elwood Henneman     &&  埃尔伍德$\cdot$亨尼曼  \\
	
	\midrule
	Emboliform nucleus     &&  栓状核  \\
	
	\midrule
	Emetine     &&  吐根碱  \\
	
	\midrule
	embryonic stem  (ES)   &&  胚胎干细胞  \\
	
	\midrule
	Emerin   &&  伊默菌素  \\
	
	\midrule
	Emery-Dreifuss muscular dystrophy   &&  埃默里-德赖弗斯肌营养不良  \\
	
	\midrule
	Emil du	Bois-Reymond   &&  埃米尔$\cdot$杜$\cdot$博伊斯-雷蒙德  \\
	
	\midrule
	Emil Kraepelin   &&  \href{https://baike.baidu.com/item/%E5%9F%83%E7%B1%B3%E5%B0%94%C2%B7%E5%85%8B%E9%9B%B7%E4%BD%A9%E6%9E%97/6486796}{埃米尔$\cdot$克雷佩林}  \\
	
	\midrule
	emotional valence     &&  情绪效价  \\
	
	\midrule
	Endel Tulving     &&  安道尔$\cdot$图威  \\
	
	\midrule
	endocytic traffic     &&  胞吞运输  \\
	
	\midrule
	Endoplasmic reticulum     &&  内质网  \\
	
	\midrule
	Endorphin     &&  内啡肽  \\
	
	\midrule
	End-foot     &&  终足  \\
	
	\midrule
	Engrailed     &&  齿状基因片段  \\
	
	\midrule
	engram     &&  记忆的痕迹  \\
	
	\midrule
	enhancer box (E-box)    &&  增强盒  \\
	
	\midrule
	entorhinal cortex     &&  内嗅皮层  \\
	
	\midrule
	Ephrins     &&  轴突导向因子  \\
	
	\midrule
	epicritic system     &&  精细觉系统  \\
	
	\midrule
	epidermal growth factor (EGF)    &&  表皮细胞生长因子  \\
	
	\midrule
	Epidermolysis bullosa simplex    &&  单纯性大疱性表皮松解症  \\
	
	\midrule
	Epimerase    &&  差向异构酶  \\
	
	\midrule
	Eric Gouaux    &&  埃里克$\cdot$古奥  \\
	
	\midrule
	Eric Lenneberg    &&  埃里克$\cdot$勒纳伯格  \\
	
	\midrule
	Ernst Brücke    &&  能斯特$\cdot$布吕克  \\
	
	\midrule
	Ernst equation    &&  能斯特方程  \\
	
	\midrule
	Ernst equation    &&  能斯特$\cdot$韦伯  \\
	
	\midrule
	Erwin Neher    &&  厄温$\cdot$内尔  \\
	
	\midrule
	Eustachian tube    &&  咽鼓管  \\
	
	\midrule
	Evan Eichler    &&  埃文$\cdot$艾克勒  \\
	
	\midrule
	estratetraenol (EST)   &&  雌四烯醇  \\
	
	\midrule
	ethyl methanesulfonate (EMS)    &&  甲磺酸乙酯  \\
	
	\midrule
	Eugene Aserinsky    &&  尤金$\cdot$阿瑟林斯基  \\
	
	\midrule
	excitatory interneurons  (EN)  &&  兴奋性中间神经元  \\
	
	\midrule
	Excitatory postsynaptic currents (EPSCs)     &&  兴奋性突触后电流  \\
	
	\midrule
	Excitatory PostSynaptic Potential (EPSP)     &&  兴奋性突触后电位  \\
	
	\midrule
	Excited-Excited neuron (EE neuron)     &&  双耳兴奋神经元  \\
	
	\midrule
	Excited-Inhibited neuron (EI neuron)     && 兴奋-抑制神经元   \\
	
	\midrule
	extensor digitorum (ED)     && 伸指肌   \\
	
	\midrule
	external globus pallidus (GPe)     && 外侧苍白球   \\
	
	\midrule
	External segment     && 外侧部   \\
	
	\midrule
	extorsion     && 外旋   \\
	
	\midrule
	extracellular signal-regulated kinase     && 细胞外信号调节激酶   \\
	
	\midrule
	Extrastriate     && 纹外皮层   \\
	
	\midrule
	extrastriate body area  (EBA)   && 纹外体区域   \\
	
	\midrule
	extrinsic reinforcement   && 外部强化   \\
	
	\midrule
	face patches   && 面部识别块   \\
	
	% 被模型预测为负的正样本;可以称作漏报率
	\midrule
	False Negative (FN)    &&  假负  \\
	
	% 被模型预测为正的负样本;可以称作误报率
	\midrule
	False Positive (FP)    &&  假正  \\
	
	\midrule
	false-positive rate (FPR)    &&  假阳性率  \\
	
	% 小脑
	\midrule
	fastigial nucleus (FN)     &&  顶核  \\
	
	\midrule
	Fechner     &&  费希纳  \\
	
	\midrule
	Felix Bloch     &&  费利克斯$\cdot$布洛赫  \\
	
	\midrule
	Fergus Craik     &&  弗格斯$\cdot$克雷克  \\
	
	\midrule
	Fernando Tello-Muñóz     &&  费尔南多$\cdot$特略-穆诺兹  \\
	
	\midrule
	fiber tract     &&  纤维束  \\
	
	\midrule
	fibroblast growth factor (FGF)    &&  成纤维细胞生长因子  \\
	
	\midrule
	Fibrous astrocyte     &&  纤维性星形胶质细胞  \\
	
	\midrule
	\makecell[l]{field excitatory \\postsynaptic potential (fEPSP)}     &&  场兴奋性突触后电位  \\
	
	\midrule
	fimbria     &&  海马伞  \\
	
	\midrule
	firing field     &&  激活野  \\
	
	\midrule
	firing rate     &&  激活率  \\
	
	\midrule
	fixation point     &&  注视点  \\
	
	\midrule
	flexor carpi ulnaris (FCU)     &&  尺侧腕屈肌  \\
	
	\midrule
	Flocculonodular lobe     &&  绒球小结叶  \\
	
	\midrule
	floor plate (FP)    &&  底板  \\
	
	\midrule
	florbetaben    &&  氟比他班  \\
	
	\midrule
	florbetapir    &&  洛贝平  \\
	
	\midrule
	Fluorescent false neurotransmitter (FFN)    &&  荧光假神经递质  \\
	
	\midrule
	flutemetamol    &&  富特米他  \\
	
	\midrule
	focal damage     &&  局灶性损伤  \\
	
	\midrule
	Focal onset     &&  局灶性起源  \\
	
	\midrule
	Focal seizure     &&  局灶性癫痫发作  \\
	
	\midrule
	follicle-stimulating hormone (FSH)     &&  卵泡刺激素  \\
	
	\midrule
	follistatin     &&  卵泡抑素  \\
	
	\midrule
	formant frequencies     &&  共振峰频率  \\
	
	\midrule
	foraging gene     &&  觅食基因  \\
	
	\midrule
	formyl peptide-related receptors (FPRs)    &&  甲酰化多肽受体  \\
	
	\midrule
	Fourth ventricle     &&  第四脑室  \\
	
	\midrule
	\makecell[l]{fragile X mental \\retardation protein (FMRP)}    &&  脆性X智力迟钝蛋白  \\
	
	\midrule
	fragile X syndrome     &&  \href{https://baike.baidu.com/item/\%E8\%84%86%E6%80%A7X%E7%BB%BC%E5%90%88%E5%BE%81/12612308}{脆性X综合症}  \\
	
	\midrule
	frameshift mutation     &&  \href{https://baike.baidu.com/item/\%E6\%A1%86%E7%A7%BB%E7%AA%81%E5%8F%98/5783764}{框移突变}  \\
	
	\midrule
	Francis Crick     &&  弗朗西斯$\cdot$克里克  \\
	
	\midrule
	Francis Galton     &&  \href{https://baike.baidu.com/item/\%E5%BC%97%E6%9C%97%E8%A5%BF%E6%96%AF%C2%B7%E9%AB%98%E5%B0%94%E9%A1%BF}{弗朗西斯$\cdot$高尔顿}  \\
	
	\midrule
	Francisco Tello     &&  弗朗西斯科$\cdot$特略  \\
	
	\midrule
	Franz Joseph Gall (Gall)     &&  弗朗兹$\cdot$约瑟夫$\cdot$加尔  \\
	
	\midrule
	Franz Kalman     &&  弗兰茨$\cdot$卡尔曼  \\
	
	\midrule
	Free nerve endings     &&  游离神经末梢  \\
	
	\midrule
	Frederic Bartlett     &&  弗雷德里克$\cdot$巴特莱特  \\
	
	\midrule
	Frederick Toates     &&  弗雷德里克$\cdot$托茨  \\
	
	\midrule
	frequency-modulated (FM)     &&  调频  \\
	
	\midrule
	Frey     &&  弗雷  \\
	
	\midrule
	Frontal Eye Field (FEF)     &&  额叶视区  \\
	
	\midrule
	frontoinsular cortex (FI)     &&  前岛叶皮层  \\
	
	\midrule
	\makecell[l]{frontotemporal dementia with \\Parkinson disease type 17 (FTPD17)}     &&  \makecell[l]{额颞叶痴\\呆伴帕金森病17型}  \\
	
	\midrule
	fruitless (Fru)     &&  无效基因  \\
	
	\midrule
	functional electrical stimulation (FES)     &&  功能性电刺激  \\
	
	\midrule
	\makecell[l]{functional magnetic \\resonance imaging (fMRI)}     &&  功能性核磁共振成像  \\
	
	\midrule
	Fukuyama congenital muscular dystrophy     &&  福山型先天性肌营养不良  \\
	
	\midrule
	fusiform face area (FFA)     &&  梭状回面孔区  \\
	
	\midrule
	fusiform gyrus (FG)     &&  梭状回  \\
	
	\midrule
	future receptive field (FRF)    &&  未来感受野  \\
	
	\midrule
	Fuxe     &&  富克塞  \\
	
	\midrule
	Fyodor Dostoyevsky     &&  \href{https://baike.baidu.com/item/%E8%B4%B9%E5%A5%A5%E5%A4%9A%E5%B0%94%C2%B7%E7%B1%B3%E5%93%88%E4%BC%8A%E6%B4%9B%E7%BB%B4%E5%A5%87%C2%B7%E9%99%80%E6%80%9D%E5%A6%A5%E8%80%B6%E5%A4%AB%E6%96%AF%E5%9F%BA}{费奥多尔$\cdot$陀思妥耶夫斯基}  \\
	
	\midrule
	F. A. Gibbs     &&  吉布斯  \\
	
	\midrule
	F. Barbara Hughes     &&  芭芭拉$\cdot$休斯  \\
	
	\midrule
	F-Actin     &&  纤维状肌动蛋白  \\
	
	\midrule
	G protein-coupled receptors    &&  G-蛋白偶联受体  \\
	
	\midrule
	\makecell[l]{G protein–gated \\ inward-rectifier K+ (GIRK)}    &&  \makecell[l]{G蛋白门控的内向整流钾}  \\
	
	\midrule
	GABA transporter    &&  $\gamma$-氨基丁酸转运蛋白  \\
	
	\midrule
	GABAergic    &&  $\gamma$-氨基丁酸能  \\
	
	\midrule
	gain field   &&  增益场  \\
	
	\midrule
	galanin (GAL)   &&  甘丙肽  \\
	
	\midrule
	gap-junction channel    &&  间隙连接通道  \\
	
	\midrule
	Garrett Alexander    &&  加勒特$\cdot$亚历山大  \\
	
	\midrule
	gastrocnemius (GAS)    &&  腓肠肌  \\
	
	\midrule
	Generalized anxiety disorder (GAD)     &&  广泛性焦虑障碍  \\
	
	\midrule
	\makecell[l]{generalized epilepsy with febrile \\ seizures plus (GEFS+ syndrome)}     &&  \makecell[l]{全面性癫痫伴热性\\惊厥附加综合症}  \\
	
	\midrule
	\makecell[l]{genetic markers of \\calcium transients (GCaMPs)}   &&  钙瞬变的遗传标记  \\
	
	\midrule
	\makecell[l]{genetically encoded voltage \\indicators (GEVIs)}   &&  基因编码的电压指示剂  \\
	
	\midrule
	Genital ridge     &&  生殖嵴  \\
	
	\midrule
	genome-wide association studies (GWAS)     &&  全基因组关联研究  \\
	
	\midrule
	Geoffrey Harris     &&  杰弗里$\cdot$哈里斯  \\
	
	\midrule
	Georg Wilhelm Friedrich Hegel     &&  \makecell{格奥尔格$\cdot$威廉$\cdot$\\弗里德里希$\cdot$黑格尔}  \\
	
	\midrule
	Georg von Békésy     &&  \href{https://baike.baidu.com/item/%E7%9B%96%E6%AC%A7%E5%B0%94%E6%A0%BC%C2%B7%E5%86%AF%C2%B7%E8%B4%9D%E5%87%AF%E5%B8%8C/8749529?fr=ge_ala}{盖欧尔格$\cdot$冯$\cdot$贝凯希}  \\
	
	\midrule
	George Berkeley     &&  乔治$\cdot$伯克利  \\
	
	\midrule
	George Eisenman     &&  乔治$\cdot$艾森曼  \\
	
	\midrule
	George Ojemann     &&  乔治$\cdot$奥杰曼  \\
	
	\midrule
	George Oliver     &&  乔治$\cdot$奥利弗  \\
	
	\midrule
	George Widener     &&  \href{https://baike.baidu.com/item/\%E4%B9%94%E6%B2%BB%C2%B7%E6%80%80%E5%BE%B7%E7%BA%B3/58006951}{乔治$\cdot$怀德纳}  \\
	
	\midrule
	gephyrin     &&  桥蛋白  \\
	
	\midrule
	Ghrelin     &&  生长激素释放肽  \\
	
	\midrule
	Gigaxonin     &&  巨轴索神经蛋白  \\
	
	\midrule
	Gill-withdrawal reflex     &&  缩鳃反射  \\
	
	\midrule
	Giulio Tononi     &&  朱利奥$\cdot$托诺尼  \\
	
	\midrule
	Giuseppe Verdi     &&  \href{https://baike.baidu.com/item/%E6%9C%B1%E5%A1%9E%E4%BD%A9%C2%B7%E5%A8%81%E5%B0%94%E7%AC%AC/292967}{朱塞佩$\cdot$威尔第}  \\
	
	\midrule
	glabrous skin && 光滑皮肤 \\
	
	\midrule
	glial cell (G)     &&  胶质细胞  \\
	
	\midrule
	\makecell[l]{glial cell line–derived \\ neurotrophic factor (GDNF) }    &&  胶质细胞源性神经营养因子  \\
	
	\midrule
	Glial scar     &&  胶质瘢痕  \\
	
	\midrule
	glial-fibrillary acidic protein (GFAP)    &&  胶质纤维酸性蛋白  \\
	
	\midrule
	Globose nucleus     && 球状核  \\
	
	\midrule
	globus pallidus (GP)    && 苍白球  \\
	
	\midrule
	Glossopharyngeal nerve     && 舌咽神经  \\
	
	\midrule
	Glucagonlike peptide-1 (GLP-1)    &&  胰高糖素样肽-1  \\
	
	\midrule
	glucocerebrosidase-1 (GBA1)     &&  葡糖脑甘酯酶-1  \\
	
	\midrule
	glucocorticoid receptor (GR)    &&  糖皮质激素受体  \\
	
	\midrule
	glutamate decarboxylase (GAD)    &&  谷氨酸脱羧酶  \\
	
	\midrule
	glutamate excitotoxicity     &&  谷氨酸兴奋性中毒  \\
	
	\midrule
	\makecell[l]{glutamate-sensing fluorescent \\ reporter (GluSnFR) }   &&  谷氨酸盐感应荧光受体  \\
	
	\midrule
	Glutamatergic neurons     &&  谷氨酸能神经元  \\
	
	\midrule
	glutamine (Gln)  &&  谷氨酰胺  \\
	
	\midrule
	glutamine synthetase  (GS)   &&  谷氨酰胺合成酶  \\
	
	\midrule
	glycine transporter (GLYT2)   &&  甘氨酸转运蛋白  \\
	
	\midrule
	Glycoprotein 130 (GP130)    &&  糖蛋白 130  \\
	
	\midrule
	goat-like     &&  	膻味  \\
	
	\midrule
	Goldman equation     &&  	戈德曼方程  \\
	
	\midrule
	Golgi apparatus     &&  	高尔基体  \\
	
	\midrule
	Golgi cell     &&  	高尔基细胞  \\
	
	\midrule
	Gonadotropin-releasing hormone (GnRH)    &&  	促性腺激素释放激素  \\
	
	\midrule
	Gordon Holmes   &&  	戈登$\cdot$霍姆斯  \\
	
	\midrule
	Gören Westling   &&  	格伦$\cdot$韦斯特林  \\
	
	\midrule
	Gracile fascicle     &&  	薄束  \\
	
	\midrule
	granule cell     &&  	颗粒细胞  \\
	
	\midrule
	Granular layer     &&  	颗粒细胞层  \\
	
	\midrule
	gravitoinertial acceleration (GIA)     &&  	重力-惯性加速度  \\
	
	\midrule
	Gregory Hickok     &&  	格雷戈里$\cdot$希科克  \\
	
	\midrule
	Gregory McCarthy     &&  	格雷戈里$\cdot$麦卡锡  \\
	
	\midrule
	Grendel     &&  	\href{https://baike.baidu.com/item/%E6%A0%BC%E4%BC%A6%E5%BE%B7%E5%B0%94/10677654}{格伦德尔}  \\
	
	\midrule
	grid cell     &&  	网格细胞  \\
	
	\midrule
	Grid field     &&  	网格野  \\
	
	\midrule
	growth associated protein 43  (GAP-43)   &&  神经生长相关蛋白-43  \\
	
	\midrule
	\makecell[l]{Growth hormone release-inhibiting \\hormone (GIH, GHRIH)}  &&  生长抑素  \\
	
	\midrule
	growth hormone–releasing hormone (GHRH)  &&  生长激素-释放激素  \\
	
	\midrule
	Growth hormone (GH)  &&  生长激素  \\
	
	\midrule
	\makecell[l]{GTP-bindingprotein (G protein,\\ GTP, Guanosine triphosphate)}     &&  三磷酸鸟苷结合蛋白  \\
	
	\midrule
	Guanine (G)     &&  鸟嘌呤  \\
	
	\midrule
	guanosine diphosphate (GDP)    &&  二磷酸鸟苷  \\
	
	\midrule
	guanylate cyclase (GC)    &&  鸟苷酸环化酶  \\
	
	\midrule
	Guillain-Barré syndrome   &&  格林-巴利综合症  \\
	
	\midrule
	Guntram Kommerell   &&  冈特拉姆$\cdot$科默累尔  \\
	
	\midrule
	Gustatory afferent nerve     &&  味觉传入神经  \\
	
	\midrule
	gustatory receptor     &&  味觉感受器  \\
	
	\midrule
	Gustav Fechner     &&  古斯塔夫$\cdot$费希纳  \\
	
	\midrule
	Gustav Fritsch     &&  古斯塔夫$\cdot$弗里奇  \\
	
	\midrule
	% 当一个物体做定轴转动时,磁矩的和角动量的大小之比
	gyromagnetic ratio     &&  磁旋比  \\
	
	\midrule
	G-actin    &&  球状肌动蛋白  \\
	
	\midrule
	Hadyn Ellis    &&  海丁$\cdot$埃利斯  \\
	
	\midrule
	Hair follicle    &&  毛囊  \\
	
	\midrule
	Håkan Olausson    &&  哈肯$\cdot$奥劳森  \\
	
	\midrule
	Haldan Keffer Hartline    &&  霍尔登$\cdot$凯弗$\cdot$哈特兰  \\
	
	\midrule
	hamstring (HAM)    &&  腘绳肌  \\
	
	\midrule
	Hans Asperger    &&  汉斯$\cdot$阿斯伯格  \\
	
	\midrule
	Hans Berger    &&  汉斯$\cdot$伯杰  \\
	
	\midrule
	Hans Kuypers    &&  汉斯$\cdot$凯珀斯  \\
	
	\midrule
	Hans Spemann    &&  汉斯$\cdot$斯佩曼  \\
	
	\midrule
	Hans-Lukas Teuber    &&  汉斯–卢卡斯$\cdot$特伯  \\
	
	\midrule
	Harlows    &&  哈洛斯  \\
	
	\midrule
	Harold Wilson    &&  \href{https://baike.baidu.com/item/%E5%93%88%E7%BD%97%E5%BE%B7%C2%B7%E5%A8%81%E5%B0%94%E9%80%8A/6406433}{哈罗德$\cdot$威尔逊}  \\
	
	\midrule
	Harry    &&  哈里  \\
	
	\midrule
	Head direction cell    &&  头朝向细胞  \\
	
	\midrule
	Hebbian plasticity    &&  赫布可塑性  \\
	
	\midrule
	hedgehog    &&  刺猬蛋白  \\
	
	\midrule
	Heiner Deubel    &&  海纳$\cdot$多伊贝尔  \\
	
	\midrule
	Heinrich Klüver    &&  海因里希$\cdot$克鲁瓦  \\
	
	\midrule
	Helen Neville    &&  海伦$\cdot$内维尔  \\
	
	\midrule
	helicotrema    &&  蜗孔  \\
	
	\midrule
	Henrik Jahnsen    &&  亨里克$\cdot$扬森  \\
	
	\midrule
	Henry Dale    &&  亨利$\cdot$戴尔  \\
	
	\midrule
	Henry Head    &&  亨利$\cdot$海德  \\
	
	\midrule
	Henry Laborit    &&  亨利$\cdot$拉博里  \\
	
	\midrule
	Henry Molaison (HM, H.M.)   &&  亨利$\cdot$莫莱森  \\
	
	\midrule
	Hensen's cells   &&  亨森细胞  \\
	
	\midrule
	Herbert Jasper     &&  赫伯特$\cdot$杰士伯  \\
	
	\midrule
	Hering-Breuer reflex     &&  黑-伯反射(黑林-伯鲁反射)  \\
	
	\midrule
	Hermann Helmholtz     &&  赫尔曼$\cdot$亥姆霍兹 \\
	
	\midrule
	Hermann von Helmholtz     &&  赫尔曼$\cdot$冯$\cdot$亥姆霍兹 \\
	
	\midrule
	herpes simplex virus (HSV)   &&  单纯疱疹病毒 \\
	
	\midrule
	heterosynaptic plasticity   &&  异突触可塑性 \\
	
	\midrule
	high vocal center (HVC)    &&  高级发声中枢  \\
	
	\midrule
	highfunctioning autism     &&  高功能孤独症  \\
	
	\midrule
	\makecell[l]{high-threshold \\mechanoreceptor (HTMR)}     &&  高阈值机械感受器  \\
	
	\midrule
	\makecell[l]{high-threshold or high-voltage\\ activated (HVA)}    &&  高阈值或高压激活  \\
	
	\midrule
	Hilde Mangold     &&  希尔德$\cdot$曼戈尔德  \\
	
	\midrule
	Hippocrates     &&  \href{https://baike.baidu.com/item/%E5%B8%8C%E6%B3%A2%E5%85%8B%E6%8B%89%E5%BA%95/180163?fr=ge_ala}{希波克拉底}  \\
	
	\midrule
	Hoffimann-reflex (H-reflex)     &&  H-反射  \\
	
	\midrule
	Home plate     &&  本垒板  \\
	
	\midrule
	homeostatic plasticity     &&  稳态可塑性  \\
	
	\midrule
	homosynaptic plasticity     &&  同突触可塑性  \\
	
	\midrule
	Homunculi     &&  小矮人  \\
	
	\midrule
	Horace Barlow     &&  霍勒斯$\cdot$巴洛  \\
	
	\midrule
	Horizontal canal     &&  水平耳道  \\
	
	\midrule
	\makecell[l]{horizontal component of the ground\\ reaction force (GRFh)}    &&  地面反作用力的水平分量  \\
	
	\midrule
	horizontal plane     &&  水平面  \\
	
	\midrule
	Horner syndrome     &&  \href{https://baike.baidu.com/item/%E9%9C%8D%E7%BA%B3%E7%BB%BC%E5%90%88%E5%BE%81/156128?fr=ge_ala}{霍纳综合症}  \\
	
	\midrule
	how pathway     &&  how 通路  \\
	
	\midrule
	Howard Curtis     &&  霍华德$\cdot$柯蒂斯  \\
	
	\midrule
	Howard Eichenbaum     &&  霍华德$\cdot$艾肯鲍姆  \\
	
	\midrule
	Huda Zoghbi     &&  胡达$\cdot$佐格比  \\
	
	\midrule
	huntingtin     &&  亨廷顿蛋白  \\
	
	\midrule
	Huntington     &&  \href{https://baike.baidu.com/item/\%E4%BA%A8%E5%BB%B7%E9%A1%BF%E7%97%85/10377104}{亨廷顿病}  \\
	
	\midrule
	Huntington-like 2     &&  类亨廷顿病2型  \\
	
	\midrule
	\makecell[l]{hydroperoxyeicosatetraenoic \\acid (HPETE)}     &&  羟过氧化二十碳四烯酸  \\
	
	\midrule
	hypercolumn     &&  \href{https://baike.baidu.com/item/%E7%9A%AE%E5%B1%82%E6%9F%B1/15899669?fr=ge_ala}{皮层柱}  \\
	
	\midrule
	\makecell[l]{hyperpolarization-activated cyclic \\nucleotide-gated channel (HCN)}     &&  \makecell[l]{超极化激活环核苷酸\\门控阳离子通道}  \\
	
	\midrule
	hypocretin     &&  下视丘分泌素  \\
	
	% XII
	\midrule
	Hypoglossal nerve     &&  舌下神经  \\
	
	\midrule
	Hypoglossal nucleus (nXIIts)    &&  舌下神经核  \\
	
	\midrule
	\makecell[l]{Hypokalemic periodic \\paralysis  (HypoPP)}  &&  低钾性周期性麻痹  \\
	
	\midrule
	hypothalamic–pituitary–adrenal (HPA)     &&  下丘脑-垂体-肾上腺  \\
	
	\midrule
	hypothalamus (HT)     &&  下丘脑  \\
	
	\midrule
	llya Repin   && 伊利亚$\cdot$列宾  \\
	
	\midrule
	Immanuel Kant   && 伊曼努尔$\cdot$康德  \\
	
	\midrule
	immunoglobulin (IgE)   && 免疫球蛋白  \\
	
	\midrule
	Immunoglobulin superfamily   && 免疫球蛋白超家族  \\
	
	\midrule
	impulses per second (ips)   && 每秒脉冲数  \\
	
	\midrule
	In vitro   && 体外  \\
	
	\midrule
	Inclusion body myositis  && 包涵体肌炎  \\
	
	\midrule
	Index finger  && 食指  \\
	
	\midrule
	induced pluripotent stem (iPS)  && 诱导的多能性干细胞  \\
	
	\midrule
	indel   && 插入缺失突变  \\
	
	\midrule
	induced pluripotent stem cells (iPSCs)  && 诱导性多功能干细胞  \\
	
	\midrule
	inferior frontal gyrus (IFG)   && 额下回  \\
	
	\midrule
	inferior oblique   && 下斜肌  \\
	
	\midrule
	\makecell[l]{inferior posterior regions of \\ prefrontal cortex (IPPFC)}  && 后下部前额叶皮层  \\
	
	\midrule
	inferior rectus   && 下直肌  \\
	
	\midrule
	inferior rectus   && 下直肌  \\
	
	\midrule
	inferotemporal cortex   && 下颞皮层  \\
	
	\midrule
	information transfer rate (ITR)   && 信息传递率  \\
	
	\midrule
	inhibitory interneuron  && 抑制性中间神经元  \\
	
	\midrule
	inhibitory postsynaptic potential (IPSP)  && 抑制性突触后电位  \\
	
	\midrule
	initial segment && 起始段  \\
	
	\midrule
	inner fiber layer (IFL)   && 内纤维层  \\
	
	\midrule
	inner subventricular zone (ISVZ)   && 室下内侧区  \\
	
	\midrule
	inositol 1,4,5-trisphosphate  (IP$_3$) && 肌醇1,4,5-三磷酸  \\
	
	\midrule
	insula (Ins)   && 脑岛  \\
	
	\midrule
	insulin-degrading enzyme (IDE)   && 胰岛素降解酶  \\
	
	\midrule
	insulin-like growth factor-1 (IGF-1)   && 胰岛素样生长因子1  \\
	
	\midrule
	Integrin  && 整联蛋白  \\
	
	\midrule
	intensity (I)  && 强度  \\
	
	\midrule
	interleukin-1 (IL-1) && 白细胞介素-1  \\
	
	\midrule
	intermediate zone (IZ)  && 中间区  \\
	
	\midrule
	Internal auditory canal   && 内耳道  \\
	
	\midrule
	Internal segment  && 内侧部  \\
	
	% 苍白球
	\midrule
	Internal segment  && 内侧部  \\
	
	\midrule
	interposed nuclei (IP)  && 间位核  \\
	
	\midrule
	interpositus nucleus  && 间位核  \\
	
	\midrule
	Interstitial fluid (ISF)  && 间质液  \\
	
	\midrule
	Intracortical electrodes   && 皮层内电极  \\
	
	\midrule
	intraparietal sulcus (IPS)   && 顶内沟  \\
	
	\midrule
	\makecell[l]{interstitial nucleus of the \\medial longitudinal fasciculus}   && 内侧纵束间质核  \\
	
	\midrule
	\makecell[l]{interstitial nucleus of the \\anterior hypothalamus  (INAH)} && 下丘脑前间质核  \\
	
	\midrule
	internal globus pallidus (GPi, Gpi)  && 苍白球内侧核  \\
	
	% 丘脑
	\midrule
	Internal medullary lamina  && 内髓板  \\
	
	\midrule
	intracortical microstimulation (ICMS)  && 皮层内微刺激  \\
	
	\midrule
	intorsion   && 内旋  \\
	
	\midrule
	Inferior frontal gyrus   && 额下回  \\
	
	\midrule
	infundibular recess (IFR)   && 漏斗隐窝  \\
	
	\midrule
	inosine monophosphate (IMP)  && 肌苷一磷酸  \\
	
	\midrule
	interaural time difference (ITD)   && 双耳时间差  \\
	
	\midrule
	intracranial electroencephalography (iEEG)  && 颅内脑电图  \\
	
	\midrule
	intralaminar thalamic nuclei (ILT)  && 丘脑髓板内核  \\
	
	\midrule
	Intelligence Quotient (IQ)   && 智商  \\
	
	\midrule
	interstitial nucleus of Cajal (iC)   && 间位核  \\
	
	\midrule
	Intrinsic Reinforcement   && 内部强化  \\
	
	\midrule
	Irving Gottesman   && 欧文$\cdot$戈特斯曼  \\
	
	\midrule
	Isidor Rabi   && 伊西多$\cdot$拉比  \\
	
	\midrule
	isoamyl acetate   && 乙酸异戊酯  \\
	
	\midrule
	Isthmic Organizer   && 组织中心  \\
	
	\midrule
	Ivan Pavlov   && 伊万$\cdot$巴甫洛夫  \\
	
	\midrule
	Jack MacMahan   && 杰克$\cdot$马克汉  \\
	
	\midrule
	Jackson Grandour   && 杰克逊$\cdot$格兰登  \\
	
	\midrule
	Jacob Schleiden   && 雅各布$\cdot$施莱登  \\
	
	\midrule
	Jacques Duchateau   && 雅克$\cdot$杜查托  \\
	
	\midrule
	JamB retinal ganglion cell (J-RGC)   && JamB 视网膜神经节细胞  \\
	
	\midrule
	James Albus   && 詹姆斯$\cdot$阿尔布斯  \\
	
	\midrule
	James Gibson   && 詹姆斯$\cdot$吉布森  \\
	
	\midrule
	James J. DiCarlo   && 詹姆斯$\cdot$迪卡罗  \\
	
	\midrule
	James J. Gibson   && 詹姆斯$\cdot$吉布森  \\
	
	\midrule
	James Olds   && 詹姆斯$\cdot$奥尔兹  \\
	
	\midrule
	James Papez   && 詹姆斯$\cdot$帕佩兹  \\
	
	\midrule
	James Rothman   && 詹姆斯$\cdot$罗斯曼  \\
	
	\midrule
	\makecell[l]{Janus kinase-signal transducer and \\activator of transcription (JAK/STAT)}  && \makecell[l]{两面神激酶-\\信号转导和转录激活因子}  \\
	
	\midrule
	Jean Pierre Changeux   && 让$\cdot$皮埃尔$\cdot$尚热  \\
	
	\midrule
	Jeffrey Friedman   && 杰弗里$\cdot$弗里德曼  \\
	
	\midrule
	Jeffrey Hall   && 杰弗里$\cdot$霍尔  \\
	
	\midrule
	Jeffrey Noebels   && 杰弗里$\cdot$诺贝尔斯  \\
	
	\midrule
	Jenny Saffran   && 珍妮$\cdot$扎弗兰  \\
	
	\midrule
	Jens Brauer   && 延斯$\cdot$布劳尔  \\
	
	\midrule
	Jerzy Konorski   && 杰泽$\cdot$科诺尔斯基  \\
	
	\midrule
	Joel Richter   && 约尔$\cdot$里希特  \\
	
	\midrule
	Johan Wessberg   && 约翰$\cdot$威斯伯格  \\
	
	\midrule
	John Carew Eccles (John C. Eccles)   && 约翰$\cdot$卡鲁$\cdot$埃克尔斯  \\
	
	\midrule
	John Dostrovsky   && 约翰$\cdot$多斯特罗夫斯基  \\
	
	\midrule
	John Langley   && 约翰$\cdot$兰利  \\
	
	\midrule
	John Locke   && 约翰$\cdot$洛克  \\
	
	\midrule
	John O’Keefe   && 约翰$\cdot$奥基夫  \\
	
	\midrule
	Jorgensen   && 约根森  \\
	
	\midrule
	Joseph Takahashi   && 约瑟夫$\cdot$高桥  \\
	
	\midrule
	jugular foramen   && 颈静脉孔  \\
	
	\midrule
	Junctional fold   && 接头皱褶  \\
	
	\midrule
	Johannes Müller   && 约翰内斯$\cdot$米勒  \\
	
	\midrule
	John B. Watson   && 约翰$\cdot$布鲁德斯$\cdot$华生  \\
	
	\midrule
	John Eccles   && 约翰$\cdot$埃克尔斯  \\
	
	\midrule
	John gardner   && 约翰$\cdot$加德纳  \\
	
	\midrule
	John Heuser   && 约翰$\cdot$霍伊泽尔  \\
	
	\midrule
	John Hughlings Jackson   && 约翰$\cdot$休林斯$\cdot$杰克逊  \\
	
	\midrule
	John Lisman   && 约翰$\cdot$利斯曼  \\
	
	\midrule
	John Marshall   && 约翰$\cdot$马歇尔  \\
	
	\midrule
	John N. Langley   && 约翰$\cdot$兰里  \\
	
	\midrule
	John Stuart Mill   && 约翰$\cdot$斯图尔特$\cdot$穆勒  \\
	
	\midrule
	John Swets   && 约翰$\cdot$斯维兹  \\
	
	\midrule
	Jonathan Wolpaw   && 乔纳森$\cdot$沃尔帕乌  \\
	
	\midrule
	Josef Rauschecker   && 约瑟夫$\cdot$罗斯柴可  \\
	
	\midrule
	Joseph Bogen   && 约瑟夫$\cdot$伯根  \\
	
	\midrule
	Juhani Hyvärinen   && 朱汉尼$\cdot$海瓦里宁  \\
	
	\midrule
	Jules Dejerine   && 朱尔斯$\cdot$代热林  \\
	
	\midrule
	Julius Axelrod   && 朱利叶斯$\cdot$阿克塞尔罗德  \\
	
	\midrule
	just noticeable difference (JND)   && 最小可觉差  \\
	
	\midrule
	Juxtaparanode   && 近结  \\
	
	\midrule
	J.Anthony Movshon   && 安东尼$\cdot$穆松  \\
	
	\midrule
	J. N. Langley   && 兰列  \\
	
	\midrule
	K complex   && K-复合波  \\
	
	\midrule
	\makecell[l]{Kalman filter decoding \\movement velocity (V-KF)}   && 解码运动速度卡尔曼滤波器  \\
	
	\midrule
	Kainate   && 红藻氨酸  \\
	
	\midrule
	Karim Nader   && 卡里姆$\cdot$奈德  \\
	
	\midrule
	Karl Lashley  && 卡尔$\cdot$拉什利  \\
	
	\midrule
	Karl-Erik Hagbarth  && 卡尔$\cdot$艾瑞克$\cdot$哈格巴斯  \\
	
	\midrule
	Kathy Cullen  && 凯西$\cdot$库伦  \\
	
	\midrule
	Kausik Si  && 考斯克$\cdot$斯伊  \\
	
	\midrule
	Kelsey Martin   && 凯尔西$\cdot$马丁  \\
	
	\midrule
	Ken Johnson   && 肯$\cdot$约翰逊  \\
	
	\midrule
	Kenneth Cole   && 肯尼思$\cdot$科尔  \\
	
	\midrule
	Kenneth Craik   && 肯尼思$\cdot$克雷克  \\
	
	\midrule
	Kent Berridge   && 肯特$\cdot$贝里奇  \\
	
	\midrule
	Kenyon cell   && 凯尼恩细胞  \\
	
	\midrule
	Kinesin motor   && 驱动蛋白  \\
	
	\midrule
	kinocilium   && 动纤毛  \\
	
	\midrule
	kisspeptin   && 亲吻肽  \\
	
	\midrule
	kiss-and-run   && 亲完就跑  \\
	
	\midrule
	Klüver-Bucy syndrome   && 克鲁瓦$\cdot$布伊综合症  \\
	
	\midrule
	Koniocellular   && 粒状  \\
	
	\midrule
	Korbinian Brodmann   && 科比尼安$\cdot$布罗德曼  \\
	
	\midrule
	Kringle domain   && 环状结构域  \\
	
	\midrule
	Kurt Koffka   && \href{https://baike.baidu.com/item/%E5%BA%93%E5%B0%94%E7%89%B9%C2%B7%E8%80%83%E5%A4%AB%E5%8D%A1/16171492}{库尔特$\cdot$考夫卡}  \\
	
	\midrule
	Lambert-Eaton syndrome   && 兰伯特-伊顿综合症(肌无力综合症)  \\
	
	\midrule
	\makecell[l]{Lambert-Eaton myasthenic \\syndrome (LEMS)} && 兰伯特-伊顿肌无力综合症  \\
	
	\midrule
	Lamin     &&  核纤层蛋白  \\
	
	\midrule
	lamina     &&  薄层  \\
	
	\midrule
	Laminin     &&  层连蛋白  \\
	
	\midrule
	Lanterman     &&  兰特曼  \\
	
	\midrule
	large diameter (Ia)     &&  大直径  \\
	
	\midrule
	Larry Squire   && 拉里$\cdot$斯奎尔  \\
	
	\midrule
	Larry Weiskrantz   && 拉里$\cdot$维斯克兰茨  \\
	
	\midrule
	Larmor equation   && 拉莫尔方程  \\
	
	\midrule
	lateral gastrocnemius (LG)  && 外侧胫骨后肌  \\
	
	\midrule
	lateral geniculate nucleus (LGN)   && 外侧膝状体核  \\
	
	\midrule
	lateral hypothalamic area (LHA)  && 下丘脑外侧区  \\
	
	\midrule
	lateral hypothalamus (LH)  && 外侧下丘脑  \\
	
	\midrule
	Lateral Inhibition  && 侧抑制  \\
	
	\midrule
	lateral intraparietal area (LIP)   && 侧顶叶  \\
	
	\midrule
	\makecell[l]{lateral intraparietal area, \\ventral portion (LIPv)}   && 侧顶叶腹侧部  \\
	
	\midrule
	lateral habenula   && 外侧缰核  \\
	
	\midrule
	lateral	hypothalamus (LH)  && 外侧下丘脑  \\
	
	\midrule
	\makecell[l]{lateral magnocellular nucleus of \\ the anterior neostriatum (LMAN)}  && \makecell[l]{新纹状体前部\\大细胞核外侧部}  \\
	
	\midrule
	lateral motor columns (LMC)   && 外侧运动柱  \\
	
	\midrule
	lateral parietal   && 顶叶外侧  \\
	
	\midrule
	lateral rectus   && 外直肌  \\
	
	\midrule
	Lateral sinus   && 横窦  \\
	
	\midrule
	lateral septum (LS)  && 外侧隔核  \\
	
	\midrule
	Lateral Superior Olivary(LSO)   && 外侧上橄榄  \\
	
	\midrule
	lateral vestibular nucleus (LVN)  && 前庭外侧核  \\
	
	\midrule
	lateral vestibulospinal tracts (LVST)  && 外前庭脊髓束  \\
	
	\midrule
	lateral view   && 侧视图  \\
	
	\midrule
	Laterodorsal tegmental nucleus   && 背外侧被盖核  \\
	
	\midrule
	Laterodorsal tegmental nucleus   && 背外侧被盖核  \\
	
	\midrule
	Laura-Anne Pettito   && 劳拉$\cdot$安妮$\cdot$佩蒂托特  \\
	
	\midrule
	Law of Effect   && \href{https://baike.baidu.com/item/%E6%95%88%E6%9E%9C%E5%BE%8B/10353079?fr=ge_ala}{效果律}  \\
	
	\midrule
	Lawrence Weiskrantz   && 劳伦斯$\cdot$魏斯克朗茨  \\
	
	\midrule
	leakage current   && 泄漏电流  \\
	
	\midrule
	left flexor motor neurons (lFmn)   && 左屈肌运动神经元  \\
	
	\midrule
	left forelimb (lFL)   && 左前肢  \\
	
	\midrule
	left hemisphere (LH)  && 左半球  \\
	
	\midrule
	left hindlimb (lHL)   && 左后肢  \\
	
	\midrule
	Levi-Montalcini   && 列维-蒙塔尔奇尼  \\
	
	\midrule
	Leo Kanner   && 利奥$\cdot$肯纳  \\
	
	\midrule
	Leslie Ungerleider   && 莱斯利$\cdot$安格莱德  \\
	
	\midrule
	Leucine-enkephalin (Leu-enkephalin)   && 亮氨酸脑啡肽  \\
	
	\midrule
	levator   && 眼提肌  \\
	
	\midrule
	Lewy body   && 路易小体  \\
	
	\midrule
	lick and groom (LG)  && 舔舐和梳理  \\
	
	\midrule
	Liddell  && 里德尔  \\
	
	\midrule
	Ligand gating  && 配体门控  \\
	
	\midrule
	Ligand-gated channel  && 配体门控通道  \\
	
	\midrule
	light period (LP)  && 光照周期  \\
	
	\midrule
	likely gene disrupting (LGD)  && 可能的基因破坏  \\
	
	\midrule
	likelihood ratio (LR) && 似然比  \\
	
	\midrule
	\makecell[l]{limb-girdle muscular \\dystrophy (LGMD)} && 肢带型肌营养不良  \\
	
	\midrule
	Limiting ridge && 限制脊  \\
	
	\midrule
	\makecell[l]{lipoprotein-related protein \\ 4 (LPR4)} && 脂蛋白相关蛋白4  \\
	
	\midrule
	Lissauer’s tract  && 背外侧束  \\
	
	\midrule
	Lloyd Jeffress  && 劳埃德$\cdot$杰夫里斯  \\
	
	\midrule
	local neuron  && 局部神经元  \\
	
	\midrule
	locus   && \href{https://baike.baidu.com/item/Locus/1628923}{基因座}  \\
	
	\midrule
	locus ceruleus (LC)  && 蓝斑  \\
	
	\midrule
	Logothetis  && 洛戈塞蒂斯  \\
	
	\midrule
	log-likelihood ratio (logLR)  && 对数似然比  \\
	
	\midrule
	long noncoding RNA (lncRNA)  && \href{https://baike.baidu.com/item/%E9%95%BF%E9%9D%9E%E7%BC%96%E7%A0%81rna/3674902}{长链非编码核糖核酸}  \\
	
	\midrule
	long-term depression (LTD)  && 长时程抑制  \\
	
	\midrule
	Long-term facilitation  && 长时程易化  \\
	
	\midrule
	Long-term memory  && 长时记忆  \\
	
	\midrule
	long-term potentiation (LTP)  && 长时程增强  \\
	
	\midrule
	Lorne Mendell  && 洛恩$\cdot$孟德尔  \\
	
	\midrule
	Louis Kunkel  && 路易斯$\cdot$孔克尔  \\
	
	\midrule
	Louis Ptácˇek  && 路易斯$\cdot$普塔切克  \\
	
	\midrule
	low voltage activated (LVA)  && 低电压激活  \\
	
	\midrule
	\makecell[l]{low-density lipoprotein \\receptor-related protein 4 (LRP4)}   && 低密度脂蛋白受体相关蛋白  \\
	
	\midrule
	low-threshold mechanoreceptors (LTMR)   && 低阈值机械感受器  \\
	
	\midrule
	low-voltage activated (LVA)   && 低压激活  \\
	
	\midrule
	Louise Goupil   && 路易丝$\cdot$古皮尔  \\
	
	\midrule
	Lynn Nadel   && 林恩$\cdot$纳德尔  \\
	
	\midrule
	lysergic acid diethylamide (LSD)  && 麦角酸二乙酰胺   \\
	
	\midrule
	Lugaro cell   && 卢加洛细胞  \\
	
	\midrule
	Luigi Galvani   && 路易吉$\cdot$加尔瓦尼  \\
	
	\midrule
	\makecell{luteinizing hormone–releasing \\hormone (LHRH)}  && 黄体生成素-释放激素  \\
	
	\midrule
	l-dihydroxyphenylalanine (l-DOPA)  && l-多巴  \\
	
	\midrule
	L.L.Thurstone  && \href{https://baike.baidu.com/item/%E7%91%9F%E6%96%AF%E9%A1%BF/9931604}{瑟斯顿}  \\
	
	\midrule
	Machado-Joseph disease   && 马查多-约瑟夫病  \\
	
	\midrule
	MacLean   && 麦克莱恩  \\
		
	\midrule
	Magnetoencephalography (MEG)   && 脑磁图  \\
	
	\midrule
	magnocellular ganglion cell   && 大神经节细胞  \\
	
	\midrule
	Mahowald   && 马霍瓦尔德  \\
	
	\midrule
	malignant hyperthermia   && 恶性高热  \\
	
	\midrule
	main olfactory bulb (MOB) && 主嗅球  \\
	
	\midrule
	main olfactory epithelium (MOE)  && 主嗅上皮  \\
	
	\midrule
	marginal zone (MZ)   && 边缘区  \\
	
	\midrule
	major depression (major depressive disorder)   && 重度抑郁症  \\
	
	\midrule
	major histocompatibility (MHC)   && 主要组织相容性  \\
	
	\midrule
	\makecell[l]{mammalian target of \\rapamycin (mTOR)}   && 哺乳动物雷帕霉素靶蛋白  \\
	
	\midrule
	\makecell[l]{mammalian target of \\rapamycin complex 1 (mTORC1)}   && \makecell[l]{哺乳动物雷帕霉素\\靶蛋白复合体1}  \\
	
	\midrule
	Mammillothalamic tract   && 乳头丘脑束  \\
	
	\midrule
	mantle shelf   && 外套膜  \\
	
	\midrule
	Marathon des Sables   && 撒哈拉沙漠马拉松赛  \\
	
	\midrule
	Marc Jeannerod   && 马克$\cdot$珍妮罗德  \\
	
	\midrule
	Marc Jeannerod   && 马库斯$\cdot$赖希勒  \\
	
	\midrule
	Margaret Harlow   && 玛格丽特$\cdot$哈洛  \\
	
	\midrule
	Mark Bear   && 马克$\cdot$贝尔  \\
	
	\midrule
	Mark Wightman   && 马克$\cdot$怀特曼  \\
	
	\midrule
	Marla Sokolowski   && 玛尔拉$\cdot$索科洛夫斯基 \\
	
	\midrule
	Martinotti cell   && 马氏细胞 \\
	
	\midrule
	Masao Ito   && 马佐$\cdot$伊托 \\
	
	\midrule
	masking effect   && 遮蔽效应 \\
	
	\midrule
	Mast cell   && 肥大细胞 \\
	
	\midrule
	\makecell[l]{Mas-related G \\protein-coupled receptor (Mrgpr)}  && \makecell[l]{Mas相关三磷酸鸟苷\\结合蛋白偶联受体} \\
	
	\midrule
	Maurice Merleau-Ponty   && 莫里斯$\cdot$梅洛-庞蒂  \\
	
	\midrule
	Maurice Smith   && 莫里斯$\cdot$史密斯  \\
	
	\midrule
	Maurizio Corbetta   && 毛里齐奥$\cdot$科尔贝塔  \\
	
	\midrule
	Mauthner cell   && 毛特讷氏细胞  \\
	
	\midrule
	Max Wertheimer   && \href{https://baike.baidu.com/item/%E9%A9%AC%E5%85%8B%E6%96%AF%C2%B7%E9%9F%A6%E7%89%B9%E6%B5%B7%E9%BB%98/16030782}{马克斯$\cdot$韦特海默}  \\
	
	\midrule
	maximal voluntary contraction (MVC)   && 最大主动收缩  \\
	
	\midrule
	maximum pulling force (MPF)   && 最大拉力  \\
	
	\midrule
	May-Britt Moser   && 梅$\cdot$布里特$\cdot$莫泽  \\
	
	\midrule
	McLeod syndrome   && 麦克劳德综合症  \\
	
	\midrule
	measles-mumps-rubella (MMR)   && 麻疹-腮腺炎-风疹  \\
	
	\midrule
	mechanical nociceptor   && 机械性伤害感受器  \\
	
	\midrule
	MECP2 duplication syndrome (MDS) && 甲基-CpG结合蛋白重复综合症  \\
	
	\midrule
	medial amygdala (MeA)  && 内侧杏仁核  \\
	
	\midrule
	\makecell[l]{medial division of the posteromedial\\ bed nucleus of the \\stria terminalis (BNSTmpm)}   && 终纹床核后侧的内侧分裂  \\
	
	\midrule
	Medial Geniculate Body (MGB)   && 内侧膝状体  \\
	
	\midrule
	medial frontal cortex (mF10)   && 内侧前额叶皮层  \\
	
	\midrule
	Medial ganglionic eminence   && 内侧神经节隆起  \\
	
	\midrule
	medial geniculate nucleus (MGN)  && 内侧膝状体核  \\
	
	\midrule
	medial intraparietal area (MIP)   && 内顶叶内区  \\
	
	\midrule
	medial lateral (ML)   && 中外侧  \\
	
	\midrule
	medial longitudinal fasciculus   && 内侧纵束  \\
	
	\midrule
	\makecell[l]{medial nucleus of the\\ dorsolateral thalamus (DLM)}  && 丘脑背外侧内侧核  \\
	
	\midrule
	\makecell[l]{Medial Nucleus of \\the Trapezoid Body (MNTB)}   && 斜方体内侧核  \\
	
	\midrule
	Medial prefrontal cortex    && 内侧前额叶皮层  \\
	
	\midrule
	Medial preoptic nucleus    && 视前内侧核  \\
	
	\midrule
	medial rectus    && 内直肌  \\
	
	\midrule
	medial reticular formation (MRF)    && 内侧网状结构  \\
	
	\midrule
	Medial Superior Olive(MSO)   && 内侧上橄榄  \\
	
	\midrule
	medial superior temporal area (MST)   && 上颞内侧区  \\
	
	\midrule
	median   && 正中神经  \\
	
	\midrule
	Median eminence   && 正中隆起  \\
	
	\midrule
	medial superior temporal (MST)   && 内侧颞叶上部  \\
	
	\midrule
	medial temporal (MT)   && 内侧颞叶  \\
	
	\midrule
	medial vestibulospinal tracts (MVST) && 内前庭脊髓束  \\
	
	\midrule
	median preoptic nucleus (MNPO, MnPO) && 视前正中核  \\
	
	\midrule
	mediodorsal nucleus (MD) && 背内侧核  \\
	
	\midrule
	medium spiny neurons && 中型棘细胞神经元 \\
	
	\midrule
	Medullary reticular formation && 延髓网状结构  \\
	
	% 触觉小体 
	% Meissnercorpuscle
	\midrule
	Meissner && 梅斯诺小体  \\
	
	\midrule
	Mel Goodale && 梅尔$\cdot$古德尔  \\
	
	\midrule
	melaninconcentrating hormone (MCH) && 黑色素浓缩素  \\
	
	\midrule
	melanocortin-4 receptors (MC4R) && 黑素皮质素受体4  \\
	
	\midrule
	Merkel cell && \href{https://baike.baidu.com/item/%E6%A2%85%E5%85%8B%E5%B0%94%E7%BB%86%E8%83%9E/10811164}{梅克尔细胞}  \\
	
	\midrule
	Merzenich && 梅策尼希  \\
	
	\midrule
	mesencephalic locomotor region (MLR)   && 中脑运动区  \\
	
	\midrule
	mesencephalic reticular formation   && 中脑网状结构  \\
	
	\midrule
	messenger Ribonucleic Acid (mRNA)   && 信使核糖核酸  \\
	
	\midrule
	metabotropic glutamate receptor (mGluR)  && 代谢型谷氨酸受体  \\
	
	\midrule
	Methionine-enkephalin   && 甲硫氨酸脑啡肽  \\
	
	\midrule
	methyl-CpG-binding protein-2 (MECP2)  && 甲基-CpG结合蛋白  \\
	
	\midrule
	Meyer's loop  && 梅耶环束 \\
	
	\midrule
	Michael Gazzaniga   && 迈克尔$\cdot$加扎尼加  \\
	
	\midrule
	Michael Mauk   && 迈克尔$\cdot$莫克  \\
	
	\midrule
	Michael Meaney   && 迈克尔$\cdot$米尼  \\
	
	\midrule
	Michael Rosbash   && 迈克尔$\cdot$罗斯巴什  \\
	
	\midrule
	Michael Shadlen   && 迈克尔$\cdot$沙德兰  \\
	
	\midrule
	Michael Young   && 迈克尔$\cdot$杨  \\
	
	\midrule
	microRNA (miRNA)   && \href{https://baike.baidu.com/item/micro\%20RNA/3683223}{微小核糖核酸}  \\
	
	\midrule
	midbrain-hindbrain boundary (MHB) && 中脑和后脑边界  \\
	
	\midrule
	middle cerebellar peduncle (MCP)  && 小脑中脚  \\
	
	\midrule
	Macrophage infiltration   && 巨噬细胞浸润  \\
	
	\midrule
	microtubule-associated proteins (MAPs)  && 微管相关蛋白  \\
	
	\midrule
	membrane time constant   && 膜时间常数  \\
	
	\midrule
	Michaelis-Menten equation   && 米氏方程  \\
	
	\midrule
	microtubule-associated protein tau    && 微管相关蛋白, tau蛋白  \\
	
	\midrule
	Microvilli    && 微绒毛  \\
	
	\midrule
	middle cingulate cortex (MCC)   && 中扣带皮层  \\
	
	\midrule
	Middle cranial fossa   && 颅中窝  \\
	
	\midrule
	Midline nuclei   && 中线核  \\
	
	\midrule
	mid-subcallosal cingulate (Mid-SCC)  && 中下胼胝体扣带皮层  \\
	
	\midrule
	mild cognitive impairment (MCI)  && 轻度认知损伤  \\
	
	\midrule
	\makecell[l]{miniature Excitatory \\PostSynaptic Potential (mEPSP) } && 微兴奋性突触后电位  \\
	
	\midrule
	missense mutation  && \href{https://baike.baidu.com/item/\%E9%94%99%E4%B9%89%E7%AA%81%E5%8F%98/4086994}{错义突变}  \\
	
	\midrule
	chondriosome hexokinase (MthK) && 线粒体已糖激酶  \\
	
	\midrule
	\makecell[l]{mitogen-activated protein kinase\\ (MAP kinase, MAPK)}   && 有丝分裂原活化蛋白激酶  \\
	
	\midrule
	\makecell[l]{mitogen-activated/ERK kinase (MEK)}   && \makecell[l]{有丝分裂原活化/\\细胞外信号调节激酶}  \\
	
	\midrule
	Mitral cell   && 僧帽细胞  \\
	
	\midrule
	Mitral cell   && 僧帽细胞  \\
	
	\midrule
	Miyoshi myopathy   && 三好氏肌肉病变  \\
	
	\midrule
	MK801   && 佐环平/地卓西平  \\
	
	\midrule
	mm Hg  && 毫米汞柱  \\
	
	\midrule
	Molecular Genetics  && \href{https://baike.baidu.com/item/%E5%88%86%E5%AD%90%E9%81%97%E4%BC%A0%E5%AD%A6/1299164?fr=ge_ala}{分子遗传学}  \\
	
	\midrule
	monoamine oxidase (MAO)   && 单胺氧化酶  \\
	
	\midrule
	Morris   && 莫里斯  \\
	
	\midrule
	Mortimer Mishkin   && 莫蒂默$\cdot$米什金  \\
	
	\midrule
	Mossy fiber   && 苔藓纤维  \\
	
	\midrule
	motor neurons (MN)   && 运动神经元  \\
	
	\midrule
	Motor proteins   && 马达蛋白  \\
	
	\midrule
	motor unit potentials (MUP)  && 运动单元电位波  \\
	
	\midrule
	Mozart  && 莫扎特  \\
	
	\midrule
	Muishkin   && 梅希金  \\
	
	\midrule
	Müllerian duct   && 副中肾管  \\
	
	\midrule
	Multiform layer   && 多形细胞层  \\
	
	\midrule
	Multiple sclerosis   && 多发性硬化  \\
	
	\midrule
	multivariate pattern analysis (MVPA)  && 多元模式分析  \\
	
	\midrule
	muscle tone   && 肌张力  \\
	
	\midrule
	\makecell[l]{muscle-specific trk-related receptor \\with a kringle domain(MuSK)}   && 跨膜受体蛋白酪氨酸激酶  \\
	
	\midrule
	Mushroom body   && 蕈体  \\
	
	\midrule
	Müllerian inhibiting substance (MIS)   && 副中肾管抑制物  \\
	
	\midrule
	Müller's muscle   && 米勒肌  \\
	
	\midrule
	myasthenia gravis (MG)    && 重症肌无力   \\
	
	\midrule
	myelin-associated glycoprotein (MAG)     && 髓磷脂相关糖蛋白   \\
	
	\midrule
	mygdala     && 杏仁核   \\
	
	\midrule
	Myelinated axon     && 有髓轴突   \\
	
	\midrule
	myelin basic protein (MBP)     && 髓磷脂碱性蛋白   \\
	
	\midrule
	Myosin heavy chain (MHC)    && 肌球蛋白重链   \\
	
	\midrule
	Myotonic dystrophy    && 强直性肌营养不良   \\
	
	\midrule
	Myotonin kinase    && 肌強直蛋白激酶   \\
	
	\midrule
	Myotubular myopathy    && 肌管性肌病   \\
	
	\midrule
	Myotubularin    && 肌微管素   \\
	
	\midrule
	Nairan Ramirez-Esparza    && 奈兰$\cdot$·拉米雷斯-埃斯帕扎   \\
	
	\midrule
	Naja Ferjan Ramirez   &&  纳亚$\cdot$费金$\cdot$拉米雷斯 \\
	
	\midrule
	Nancy Kanwisher   &&  南希$\cdot$坎韦施 \\
	
	\midrule
	Nathaniel Kleitman   &&  纳撒尼尔$\cdot$克莱特曼 \\
	
	\midrule
	Neal Cohen   &&  尼尔$\cdot$科恩 \\
	
	\midrule
	Nebulin   &&  伴肌动蛋白 \\
	
	\midrule
	Nemaline rod myopathy   &&  杆状体肌病 \\
	
	\midrule
	nebulin   &&  伴肌动蛋白 \\
	
	\midrule
	Needle electrode   &&  针状电极 \\
	
	\midrule
	Neoendorphin   &&  新内啡肽 \\
	
	\midrule
	nerve growth factor (NGF)   &&  神经生长因子 \\
	
	\midrule
	Netrin   &&  轴突导向因子 \\
	
	\midrule
	neurexins   &&  神经外素 \\
	
	\midrule
	neurofilament heavy polypeptide (NFH)   &&  神经丝重多肽 \\
	
	\midrule
	neurokinin-1 (NK1) receptor   && 神经激肽受体-1 \\
	
	\midrule
	neuroligin 4X (NLGN4X)  && 神经连接蛋白4X \\
	
	\midrule
	Neuroligins (NLs)   && 神经连接蛋白 \\
	
	\midrule
	neurotransmitter sodium symporter (NSS)  && 神经递质钠转运体 \\
	
	\midrule
	neurotrophins (NT)   && 神经营养因子 \\
	
	\midrule
	New World monkey   && 新大陆猴 \\
	
	\midrule
	neural cell adhesion molecule (NCAM)  && 神经细胞粘附分子 \\
	
	\midrule
	Neuroendocrine cell   && 神经内分泌细胞 \\
	
	\midrule
	Neurogenin   && 神经元素 \\
	
	\midrule
	Neuropathic pain   && 神经病理性疼痛 \\
	
	\midrule
	neuropeptide receptor (npr)   && 神经肽受体 \\
	
	\midrule
	neuropeptide Y (NPY)  && 神经肽Y \\
	
	\midrule
	neuropil  && 神经胶质细胞 \\
	
	\midrule
	nicotinamide adenine dinucleotide (NADH)  && 烟酰胺腺嘌呤二核苷酸 \\
	
	\midrule
	\makecell[l]{nicotinamide mononucleotide \\ adenyltransferase 1 (NMNAT1)}  && \makecell[l]{烟酰胺单核苷酸\\腺嘌呤转移酶1} \\
	
	\midrule
	nicotinic receptor (nic)  && 烟碱受体 \\
	
	\midrule
	Nigel Unwin   && 奈杰尔$\cdot$昂温 \\
	
	\midrule
	Nils Hillarp   && 尼尔斯$\cdot$希勒 \\
	
	\midrule
	Nima Ghitani   && 尼姆$\cdot$基塔尼 \\
	
	\midrule
	Nissl substance   && 尼氏体 \\
	
	\midrule
	nitric oxide (NO)   && 一氧化氮 \\
	
	\midrule
	Noam Chomsky   && 诺姆$\cdot$乔姆斯基 \\
	
	\midrule
	Node of Ranvier   && 郎飞结 \\
	
	\midrule
	noggin   && 人头蛋白 \\
	
	\midrule
	Nogo   && 勿动蛋白 \\
	
	\midrule
	Nonassociative learning   && 非联想学习 \\
	
	\midrule
	nonsense mutation   && \href{https://baike.baidu.com/item/%E6%97%A0%E4%B9%89%E7%AA%81%E5%8F%98/4087071}{无义突变} \\
	
	\midrule
	\makecell[l]{nonsteroidal anti-inflammatory \\ drugs (NSAIDs)}   && 非甾体抗炎药 \\
	
	\midrule
	Non-coding RNA (ncRNA)   && \href{https://baike.baidu.com/item/%E9%9D%9E%E7%BC%96%E7%A0%81RNA/10066623}{非编码核糖核酸} \\
	
	\midrule
	\makecell[l]{non-voltage-activated sodium \\leak nonselective (NALCN)}  && \makecell[l]{非电压激活钠\\泄漏非选择性} \\
	
	\midrule
	noradrenaline   && 去甲肾上腺素 \\
	
	\midrule
	noradrenergic (NA)   && 去甲肾上腺素能 \\
	
	\midrule
	noradrenergic neurons (A1)  && 去甲肾上腺素能神经元 \\
	
	\midrule
	norepinephrine (NE)   && 去甲肾上腺素 \\
	
	\midrule
	norepinephrine transporter (NET)   && 去甲肾上腺素转运蛋白 \\
	
	\midrule
	Normetanephrine (NM)  && 去甲肾上腺素 \\
	
	\midrule
	Nuclear envelope   && 核被膜  \\
	
	\midrule
	nuclear import receptors   && 核输入受体  \\
	
	\midrule
	Nucleus accumbens (NAcc)  && 伏隔核  \\
	
	\midrule
	Nucleus ambiguus   && 疑核  \\
	
	\midrule
	nucleus of Darkshevich   && 达克谢维奇核  \\
	
	\midrule
	Nucleus of the solitary tract (NST, NTS)  && 孤束核  \\
	
	\midrule
	nucleus prepositus hypoglossi   && 舌下前置核  \\
	
	\midrule
	nucleus reticularis magnocellularis (NRMc)   && 大细胞网状核  \\
	
	\midrule
	nucleus reticularis gigantocellularis (NRGc)   && 巨细胞网状核  \\
	
	\midrule
	nucleus reticularis pontis oralis (NRPo)   && 网状脑桥嘴核  \\
	
	\midrule
	nucleus subceruleus   && 蓝斑下核  \\
	
	\midrule
	N-methyl-4-phenylpyridinium (MPP$^+$)  && N-甲基-4-苯基吡啶  \\
	
	\midrule
	N-Methyl-D-Aspartate (NMDA)   && N-甲基-D-天冬氨酸  \\
	
	\midrule
	N-terminal domain   && N-末端结构域  \\
	
	\midrule
	obsessive-compulsive disorder     && 强迫症   \\
	
	\midrule
	occipitotemporal cortex     && 枕颞皮层   \\
	
	\midrule
	oculocutaneous albinism II (OCA2)     && 2 型眼皮肤白化病   \\
	
	% III
	\midrule
	Oculomotor nerve (oculomotor)     && 动眼神经   \\
	
	\midrule
	oculomotor vermis (OMV)     && 动眼神经小脑蚓体   \\
	
	\midrule
	Oculopharyngeal dystrophy     && 眼咽型肌营养不良   \\
	
	\midrule
	Odorant     && 气味剂   \\
	
	\midrule
	OFF Cell     && 撤光细胞   \\
	
	\midrule
	FMB midget bipolar (IMB)   && 撤光侏儒双极细胞   \\
	
	\midrule
	Old World monkeys     && 旧大陆猴   \\
	
	\midrule
	Olfactory bulb     && 嗅球   \\
	
	\midrule
	olfactory cortex     && 嗅觉皮层   \\
	
	\midrule
	Olfactory epithelium     && 嗅上皮   \\
	
	\midrule
	olfactory marker protein     && 嗅觉标记蛋白   \\
	
	\midrule
	Olfactory sensory neurons     && 嗅觉感受神经元   \\
	
	\midrule
	olfactory tract     && 嗅束   \\
	
	\midrule
	olfactory tubercle     && 嗅结节   \\
	
	\midrule
	Oligodendrocyte     && 少突细胞   \\
	
	\midrule
	\makecell[l]{oligodendrocyte-myelin \\glycoprotein (OMgp)}    && 髓鞘少突胶质细胞糖蛋白   \\
	
	\midrule
	omnipause neurons    && 全面停止神经元   \\
	
	% 更喜欢光明
	\midrule
	ON Cell    && 给光细胞   \\
	
	\midrule
	ON midget bipolar (IMB)   && 给光侏儒双极细胞   \\
	
	\midrule
	Onuf’s nucleus (Onufrowicz nucleus)   && 奥奴弗罗维奇核   \\
	
	\midrule
	oocyte     && \href{https://baike.baidu.com/item/%E5%8D%B5%E6%AF%8D%E7%BB%86%E8%83%9E}{卵母细胞}   \\
	
	\midrule
	opposite direction (OD)     && 反向   \\
	
	\midrule
	optic chiasm (OC)     && 视交叉   \\
	
	\midrule
	Optic foramen     && 视神经孔   \\
	
	\midrule
	optic nerve     && 视神经   \\
	
	\midrule
	optimal linear estimator (OLE)    && 最佳线性估计器   \\
	
	\midrule
	optokinetic response    && 眼动反应   \\
	
	\midrule
	orbital frontal cortex (OFC, OF11)   && 眶额皮层 \\
	
	\midrule
	orexin  && 食欲素 \\
	
	\midrule
	Orexinergic Neurons  && 促食欲素能神经元 \\
	
	\midrule
	Orphan receptor     && \href{https://baike.baidu.com/item/%E5%AD%A4%E5%84%BF%E5%8F%97%E4%BD%93/8642007?fr=ge_ala}{孤儿受体}   \\
	
	\midrule
	Orphanin FQ     && 孤啡肽   \\
	
	\midrule
	Ortrud Steinlein     && 奥特鲁$\cdot$施泰因莱因   \\
	
	\midrule
	Oswald Steward     && 奥斯瓦尔德$\cdot$斯图尔德   \\
	
	\midrule
	otoferlin    && 耳畸蛋白   \\
	
	\midrule
	Otto Loewi     && 奥托$\cdot$勒维   \\
	
	\midrule
	outer fiberlayer (OFL)     && 外纤维层   \\
	
	\midrule
	outer subventricular zone (OSVZ)     && 室下外侧区   \\
	
	\midrule
	outsider artist     && 世外艺术家   \\
	
	\midrule
	owl monkey     && 夜猴   \\
	
	\midrule
	oxygen saturation (SaO2)     && 血氧饱和度   \\
	
	\midrule
	oxytocin (OXY)    && 催产素	   \\
	
	\midrule
	P element  && P 元件   \\
	
	\midrule
	Pablo Picasso  && \href{https://baike.baidu.com/item/%E5%B7%B4%E5%8B%83%E7%BD%97%C2%B7%E6%AF%95%E5%8A%A0%E7%B4%A2/22027443}{巴勃罗$\cdot$毕加索}   \\
	
	\midrule
	Pacinian corpuscle  && 环层小体, 帕西尼安小球   \\
	
	\midrule
	Paired helical filaments  && 双螺旋丝   \\
	
	\midrule
	paired-association task && 配对偶联任务   \\
	
	\midrule
	Papio papio  && 几内亚狒狒   \\
	
	\midrule
	parabrachial nucleus (PB) && 臂旁核   \\
	
	\midrule
	parafacial zone (PFZ) && 面神经旁核   \\
	
	\midrule
	parafascicular thalamic nucleus && 丘脑束旁核 \\
	
	\midrule
	parahippocampal cortex (PHC)  && 海马旁回   \\
	
	\midrule
	parahippocampal place area (PPA)  && 海马旁回   \\
	
	\midrule
	parahippocampal gyrus  (Ph)   && 海马旁回   \\
	
	\midrule
	Parallel fiber (PF)    && 平行纤维   \\
	
	\midrule
	paramedian pontine reticular formation (PPRF)  && 脑桥旁正中网状结构   \\
	
	\midrule
	Paranodal loop (PNL)    && 帕拉节环   \\
	
	\midrule
	paraspinals (PSP)     && 椎旁肌   \\
	
	\midrule
	parasubiculum     && 旁下托   \\
	
	\midrule
	paraventricular hypothalamus     && 下丘脑室旁   \\
	
	\midrule
	Paraventricular nucleus     && 旁室核   \\
	
	\midrule
	\makecell[l]{paraventricular nucleus of\\ the hypothalamus (PVH)}    && 下丘脑室旁核   \\
	
	\midrule
	Paraxial mesoderm     && 轴旁中胚层   \\
	
	\midrule
	parent-child trios     && 亲子三人组   \\
	
	\midrule
	parietal areas (PE)    && 顶叶   \\
	
	\midrule
	Parietal reach region (PRR)     && 顶叶到达区   \\
	
	\midrule
	parietal rostroventral cortex (PR)     && 顶叶头腹侧皮层   \\
	
	\midrule
	parietal ventral cortex (PV)     && 顶叶腹侧皮层   \\
	
	\midrule
	parieto-insular vestibular cortex (PIVC)     && 顶-岛前庭皮层   \\
	
	\midrule
	parieto-occipital sulcus area 2  (POS2)   && 顶枕沟2区   \\
	
	\midrule
	Parkinson disease (PD)   && 帕金森病   \\
	
	\midrule
	paroxysmal depolarizing shift (PDS)     && 阵发性去极化漂移   \\
	
	\midrule
	parvalbumin  (PV)   && 小清蛋白   \\
	
	\midrule
	Patrick Haggard     && 帕特里克$\cdot$哈格德   \\
	
	\midrule
	Patrick Wall     && 帕特里克$\cdot$沃尔   \\
	
	\midrule
	Paul Bucy     && 保罗$\cdot$布西  \\
	
	\midrule
	Paul Ehrlich     && 保罗$\cdot$埃尔利希  \\
	
	\midrule
	Paul Fatt     && 保罗$\cdot$法特  \\
	
	\midrule
	Paul Greengard     && 保罗$\cdot$格林加德  \\
	
	\midrule
	Paul Hoffmann     && 保罗$\cdot$霍夫曼  \\
	
	\midrule
	Paul Iverson     && 保罗$\cdot$艾弗森  \\
	
	\midrule
	Paul Lauterbur     && 保罗$\cdot$劳特布尔  \\
	
	\midrule
	Paul MacLean     && 保罗$\cdot$麦克莱恩  \\
	
	\midrule
	Paul Mueller     && 保罗$\cdot$穆勒  \\
	
	\midrule
	Paul Pierre Broca     && 皮埃尔$\cdot$保尔$\cdot$布罗卡  \\
	
	\midrule
	Paul Weiss     && 保罗$\cdot$韦斯  \\
	
	\midrule
	Paul Wender     && 保罗$\cdot$文德  \\
	
	\midrule
	PE intraparietal area (PEip)   && 顶内区   \\
	
	\midrule
	pectoralis (Pec)    && 胸肌   \\
	
	\midrule
	pedunculopontine nucleus (PPN)     && 脑桥脚核   \\
	
	\midrule
	Peltier     && 帕尔贴   \\
	
	\midrule
	\makecell[l]{pentameric ligand-gated \\ion channels (pLGIC)}     && \makecell[l]{五聚体配体\\门控离子通道}   \\
	
	\midrule
	peptide YY (PYY)     && 多肽 YY   \\
	
	\midrule
	PER protein     && 周期蛋白   \\
	
	\midrule
	per gene (per)     && 节律基因   \\
	
	\midrule
	\makecell[l]{percentage of maximum \\score possible (POMP)}     && 可能的最大分数百分比   \\
	
	\midrule
	perceptual constancy     && \href{https://baike.baidu.com/item/%E7%9F%A5%E8%A7%89%E6%81%92%E5%B8%B8%E6%80%A7}{知觉恒常性}   \\
	
	\midrule
	perceptual null point     && 感知零点   \\
	
	\midrule
	Perforant pathway     && 穿通通路   \\
	
	\midrule
	periaqueductal gray matter (PAG)    && 中脑导水管周围灰质   \\
	
	\midrule
	Periaxin    && 轴周蛋白   \\
	
	\midrule
	Periglomerular cell     && 球周细胞   \\
	
	\midrule
	Perineural sheath     && 神经鞘   \\
	
	\midrule
	Period genes (Per)    && 周期基因   \\
	
	\midrule
	peripheral myelin protein 22  (PMP22)   && 外周鞘磷脂蛋白22   \\
	
	\midrule
	peristimulus time histogram   && 刺激时间直方图   \\
	
	\midrule
	perisylvian language area (PSL)   && 外侧裂周语言区   \\
	
	\midrule
	Peter Agre     && 彼得$\cdot$阿格雷   \\
	
	\midrule
	Peter Eimas     && 彼得$\cdot$艾马斯   \\
	
	\midrule
	Peter Halligan     && 彼得$\cdot$哈里根   \\
	
	\midrule
	Peter Mansfield     && 彼得$\cdot$曼斯菲尔德   \\
	
	\midrule
	Peter Strick     && 彼得$\cdot$斯特里克   \\
	
	\midrule
	petit mal     && 癫痫小发作   \\
	
	\midrule
	Petrous temporal bone     && 颞骨岩   \\
	
	\midrule
	phencyclidine  (angel dust, PCP)  &&  苯环利定, 天使尘   \\
	
	\midrule
	Phenoxybenzamine  &&  酚苄明   \\
	
	\midrule
	Phenylketonuria (PKU)    &&  苯丙酮尿(苯丙酮尿症)   \\
	
	\midrule
	Philip Bard     && 菲利普$\cdot$巴德   \\
	
	\midrule
	Phineas Gage     && 菲尼斯$\cdot$盖奇   \\
	
	\midrule
	Phosphatase and tensin homolog (PTEN)    && 蛋白酪氨酸磷酸酶基因   \\
	
	\midrule
	phosphate-activated glutaminase (PAG)   && 磷酸激活的谷氨酰胺酶   \\
	
	\midrule
	phosphatidylinositol (PI)    && 磷脂酰肌醇   \\
	
	\midrule
	\makecell[l]{phosphatidylinositol \\ 4,5-bisphosphate ($ PIP_2 $)}    && 磷脂酰肌醇-4,5-二磷酸   \\
	
	\midrule
	phosphatidylinositol-3 kinase (PI3-K)   && 磷酸肌醇3-激酶   \\
	
	\midrule
	phospholipase C (PLC)     && 磷脂酶C   \\
	
	\midrule
	phosphoprotein phosphatase 1 (PP1)     && 磷蛋白磷酸酶1   \\
	
	\midrule
	photoreceptors     && 光感受器   \\
	
	\midrule
	physiognomy     && 相面术   \\
	
	\midrule
	Pial surface     && 软脑膜表面   \\
	
	\midrule
	Pierre Broca     && 皮埃尔$\cdot$布罗卡   \\
	
	\midrule
	Pierre Flourens     && 皮埃尔$\cdot$弗卢龙   \\
	
	\midrule
	Pierre Fourneret     && 皮埃尔$\cdot$富尔纳雷   \\
	
	\midrule
	Ping Mamiya     && 平$\cdot$玛米亚   \\
	
	\midrule
	Pioneer neuron     && 先驱神经元   \\
	
	\midrule
	Piriform cortex     && 梨状皮层   \\
	
	\midrule
	Pittsburgh compound B     && 匹兹堡化合物B   \\
	
	\midrule
	Pituitary gland     && 垂体腺   \\
	
	\midrule
	\makecell[l]{PIWI-interacting RNA \\ (piRNA) }  && \makecell[l]{与Piwi蛋白相作用\\的核糖核酸}   \\
	
	\midrule
	pku   && 苯丙酮尿症基因   \\
	
	\midrule
	place cell   && 位置细胞   \\
	
	\midrule
	place field   && 位置野   \\
	
	\midrule
	Plectin   && 网蛋白   \\
	
	\midrule
	Plexin   && 丛蛋白   \\
	
	\midrule
	phosphodiesterase (PDE)  && 磷酸二酯酶   \\
	
	\midrule
	positive allosteric modulator   && 正变构调节剂   \\
	
	\midrule
	presubiculum   && 海马前下托   \\
	
	\midrule
	pre-Bötzinger complex   && 前包钦格复合体   \\
	
	\midrule
	primary somatosensory cortex   && 初级体感皮层   \\
	
	\midrule
	priority map   && 优先级图   \\
	
	\midrule
	\makecell[l]{PTEN-induced putative \\kinase 1 (PINK1)}     && \makecell[l]{蛋白酪氨酸磷酸酶基因\\诱导假定激酶1}   \\
	
	\midrule
	point mutation     && 点突变   \\
	
	\midrule
	Poly A binding protein     && 多聚腺苷酸结合蛋白   \\
	
	\midrule
	polygenic risk scores (PRS)     && 多基因风险评分   \\
	
	\midrule
	polymerase (Pol)     && 聚合酶   \\
	
	\midrule
	Polymodal     && 多觉型   \\
	
	\midrule
	polymodal nociceptor     && 多觉型伤害性感受器   \\
	
	\midrule
	Pontine flexure     && 桥曲   \\
	
	\midrule
	pontine micturition center  (PMC)   && 脑桥排尿中枢   \\
	
	\midrule
	positron emission tomography (PET)     && 正电子发射断层成像   \\
	
	\midrule
	posterior commissure     && 后连合   \\
	
	\midrule
	Posterior cranial fossa     && 颅后窝   \\
	
	\midrule
	Posterior Parietal Cortex (PPC)     && 后顶叶皮层   \\
	
	\midrule
	postsynaptic density (PSD)     && 突触后致密物   \\
	
	\midrule
	pontine nuclei (PN)    && 	脑桥核   \\
	
	\midrule
	pontomedullary reticular formation (PMRF)   && 	桥髓网状结构   \\
	
	\midrule
	population vector algorithm (PVA)   && 	群体向量法   \\
	
	\midrule
	\makecell[l]{Positive and Negative Affect\\ Schedule (PANAS)}     && 	正性负性情绪量表   \\
	
	\midrule
	posterior superior temporal gyri     && 	颞上回后部   \\
	
	\midrule
	Postganglionic neurons     && 	节后神经元   \\
	
	\midrule
	posttraumatic stress disorder (PTSD)     && 	创伤后应激障碍   \\
	
	\midrule
	potassium-aggravated myotonia     && 	钾加重性肌强直   \\
	
	% 普拉德$\cdot$威利综合症
	\midrule
	Prader-Willi syndromes     && 	\href{https://baike.baidu.com/item/\%E5%B0%8F%E8%83%96%E5%A8%81%E5%88%A9%E7%97%87/7472495}{小胖威利症}   \\
	
	\midrule
	Prechordal plate     && 	脊索前板   \\
	
	\midrule
	predorsal premotor cortex  (pre-PMd)    && 	背前前运动皮层   \\
	
	\midrule
	prefrontal cortex (F46)     && 	前额叶皮层   \\
	
	\midrule
	Preganglionic neurons     && 	节前神经元   \\
	
	\midrule
	Preganglionic autonomic motor neurons (PGC)     && 	节前自主运动神经元   \\
	
	\midrule
	Premature stop codon     && 	提前终止密码子   \\
	
	% 下丘脑
	\midrule
	preoptic area (POA)     && 	视前区   \\
	
	\midrule
	Presenilin-1 (PS1)     && 	早老蛋白-1   \\
	
	\midrule
	prestin     && 	快蛋白   \\
	
	\midrule
	presupplementary motor area (pre-SMA)    && 	前辅助运动区   \\
	
	\midrule
	Presynaptic terminal     && 	突触前末梢   \\
	
	\midrule
	Pretectum     && 	前顶盖   \\
	
	\midrule
	pre-supplementary motor area (Pre-SMA)     && 	前辅助运动区   \\
	
	\midrule
	Primary active transport   && 初级主动运输  \\
	
	% 小脑
	\midrule
	Primary fissure (PF)   && 原裂  \\
	
	\midrule
	primary motor cortex (M1)   && 初级运动皮层  \\
	
	\midrule
	primary somatosensory cortex (S-I)   && 初级躯体感觉皮层  \\
	
	% 启动效应是指快速呈现的刺激(启动刺激)对随后出现的第二个刺激(目标刺激)的处理产生的积极或消极的影响
	\midrule
	Priming   && 启动  \\
	
	\midrule
	principle of dynamic polarization   && 动态极化原理  \\
	
	\midrule
	Principles of Psychology   && 《心理学原理》  \\
	
	\midrule
	Prolactin release-inhibiting hormone (PIH)  && 催乳素分泌抑制因子  \\
	
	\midrule
	prodynorphin (PDYN)  && 前强啡肽  \\
	
	\midrule
	proenkephalin (PENK)  && 前脑啡肽  \\
	
	\midrule
	profilin  && 抑制蛋白  \\
	
	\midrule
	progenitor cell (P)   && 祖细胞  \\
	
	\midrule
	Projection interneuron   && 投射中间神经元  \\
	
	\midrule
	proneurotrophins   && 神经营养素  \\
	
	\midrule
	proopiomelanocortin (POMC)  && 前阿黑皮素  \\
	
	\midrule
	Prostaglandin   && 前列腺素  \\
	
	\midrule
	Prostigmin   && 新斯的明  \\
	
	\midrule
	prosopagnosia   && 面孔失认症  \\
	
	\midrule
	protein kinase A (PKA)   && 蛋白激酶A  \\
	
	\midrule
	protein kinase C (PKC)   && 蛋白激酶C  \\
	
	\midrule
	protein kinase M$\zeta$ (PKM$\zeta$)   && 蛋白激酶M$\zeta$  \\
	
	\midrule
	\makecell[l]{protein-O-mannosyl \\ transferase 1 (POMT1)}     && 蛋白O-甘露糖基转移酶1   \\
	
	\midrule
	\makecell[l]{protein-Omannosyl $\alpha$-,\\2-N-acetylglucosaminyl transferase \\(POMGnT1)}     && \makecell[l]{N-乙酰氨基葡萄糖-\\甘露糖转移酶1}  \\
	
	\midrule
	proteolipid protein (PLP)   && 蛋白脂蛋白  \\
	
	\midrule
	protocadherin (Pcdh)  && 原钙粘蛋白  \\
	
	\midrule
	protopathic system   && 粗感觉系统  \\
	
	\midrule
	Proximal myotonic dystrophy   && 近端肌强直性肌病  \\
	
	\midrule
	pruritogens  && 致痒素  \\
	
	\midrule
	Rapsyn   && 突触受体相关蛋白  \\
	
	\midrule
	Posttetanic potentiation   && 强直刺激后增强  \\
	
	\midrule
	primary auditory cortex (A1)   && 初级听觉皮层  \\
	
	\midrule
	prolactin  (PRL)  && 催乳素  \\
	
	\midrule
	pro-opiomelanocortin (POMC)  && 前阿黑皮素  \\
	
	\midrule
	pS  && 皮西门子  \\
	
	\midrule
	pteridine (Pt)  && \href{https://baike.baidu.com/item/%E8%9D%B6%E5%95%B6/5306574?fr=ge_ala}{蝶啶}  \\
	
	\midrule
	pulses per second (pps)  && 每秒脉冲数  \\
	
	\midrule
	Pulvinar   && 丘脑枕  \\
	
	\midrule
	pulvinar nucleus (PL)  && 枕核  \\
	
	\midrule
	Purkinje cell (PC)   && 浦肯野细胞  \\
	
	\midrule
	putamen (Put)   && 壳核  \\
	
	\midrule
	pyloric (PY)  && 幽门  \\
	
	\midrule
	pyloric dilator (PD)  && 幽门扩张器  \\
	
	\midrule
	Pyramidal decussation   && 锥体交叉  \\
	
	\midrule
	pyramidal neurons   && 锥体神经元  \\
	
	\midrule
	quadriceps (QUAD)   && 四头肌  \\
	
	\midrule
	\makecell[l]{quantitative magnetic resonance \\imaging (qMRI)} && 定量磁共振成像  \\
	
	\midrule
	radiofrequency (RF)  && 射频  \\
	
	\midrule
	Ragnar Granit   && 拉格纳$\cdot$格拉尼特  \\
	
	% 用于连接另外两个神经的神经
	\midrule
	Ramus communicantes   && 连通分支  \\
	
	\midrule
	Ranulfo Romo   && 拉努尔福$\cdot$罗莫  \\
	
	\midrule
	Randy Flanagan   && 兰迪$\cdot$弗拉纳根  \\
	
	\midrule
	raphe nuclei   && 中缝核  \\
	
	\midrule
	rapid eye movement (REM)   && 快速眼动  \\
	
	\midrule
	rapidly adapting (RA)   && 快适应  \\
	
	\midrule
	\makecell[l]{rapidly adapting low-threshold\\ mechanoreceptors (RALTMRs)}  && 快适应低阈值机械感受器  \\
	
	\midrule
	rapsyn  && 受体相连突触蛋白  \\
	
	\midrule
	rate-limiting factor  && 限速因子  \\
	
	\midrule
	rectifying synapses  && 矫正突触  \\
	
	\midrule
	Raymond Dodge  && 雷蒙德・道奇  \\
	
	\midrule
	reaction time  && 反应时间  \\
	
	\midrule
	\makecell[l]{recalibrated feedback intention-trained\\ Kalman filter (RF)}   && \makecell[l]{重新校准反馈意图\\训练卡尔曼滤波器}  \\
	
	\midrule
	receiver operating characteristic (ROC)   && 受试者工作特征  \\
	
	\midrule
	Receptive Field (RF)   && 感受野  \\
	
	\midrule
	\makecell[l]{receptor for advanced glycation \\end products (RAGE)}   && 晚期糖基化终产物受体  \\
	
	\midrule
	Receptor potential   && 受体电位  \\
	
	\midrule
	Receptor tyrosine kinase   && 受体酪氨酸激酶  \\
	
	\midrule
	rectus muscle   && 直肌  \\
	
	\midrule
	Reelin   && 络丝蛋白  \\
	
	\midrule
	\makecell[l]{regional cerebral metabolic \\rate for glucose (rCMRglc)}  && 区域大脑葡萄糖代谢率  \\
	
	\midrule
	regions of interest (ROI)   && 感兴趣区域  \\
	
	\midrule
	regulatory subunit (R)   && 调节亚基  \\
	
	\midrule
	Reissner’s membrane   && 前庭膜  \\
	
	\midrule
	relay neuron   && 中继神经元  \\
	
	\midrule
	Rene Descartes (René Descartes)  && 勒内$\cdot$笛卡尔  \\
	
	\midrule
	Renshaw cell   && 闰绍细胞  \\
	
	\midrule
	representational similarity analysis (RSA)  && 表征相似性分析  \\
	
	\midrule
	Reserpine   && 利血平  \\
	
	\midrule
	response field (RF)   && 响应场  \\
	
	\midrule
	responsive neurostimulation system (RNS)   && 反应神经刺激系统  \\
	
	\midrule
	reticular nucleus of the thalamus (RT)  && 丘脑网状核  \\
	
	\midrule
	reticulospinal tracts (RST)   && 网状脊髓束  \\
	
	\midrule
	Retinal ganglion cells (RGC)   && 视网膜神经节细胞  \\
	
	\midrule
	retinotopic map   && 视网膜脑图  \\
	
	\midrule
	retrosplenial complex (RSC)   && 压后皮层复合体  \\
	
	\midrule
	retrosplenial cortex   && 压后皮层  \\
	
	\midrule
	Rett syndrome   && \href{https://baike.baidu.com/item/\%E9%9B%B7%E7%89%B9%E9%9A%9C%E7%A2%8D/22296155}{雷特综合症}  \\
	
	\midrule
	rhodopsin (R)   && 视紫红质  \\
	
	\midrule
	rhombomere   && 菱脑节  \\
	
	\midrule
	Rhythm-generating extensor (eR)  && 产生节律的伸肌  \\
	
	\midrule
	Rhythm-generating flexor (fR)  && 产生节律的屈肌  \\
	
	\midrule
	Ribonucleic Acid   && \href{https://baike.baidu.com/item/\%E6%A0%B8%E7%B3%96%E6%A0%B8%E9%85%B8/541373}{核糖核酸}   \\
	
	\midrule
	Ribonucleic Acid interference (RNAi) &&  核糖核酸干扰  \\
	
	\midrule
	Ribosomal RNA (rRNA)   && \href{https://baike.baidu.com/item/\%E6%A0%B8%E7%B3%96%E4%BD%93RNA/3752312}{核糖体核糖核酸}  \\
	
	\midrule
	Richard Andersen   && 理查德$\cdot$安德森  \\
	
	\midrule
	Richard Bagnall   && 理查德$\cdot$巴格诺尔  \\
	
	\midrule
	Richard Ivry   && 理查德$\cdot$伊夫里  \\
	
	\midrule
	Richard L. Gregory   && 理查德$\cdot$格里高利  \\
	
	\midrule
	Richard Morris   && 理查德$\cdot$莫里斯  \\
	
	\midrule
	Richard Thompson   && 理查德$\cdot$汤普森  \\
	
	\midrule
	Richard Semon   && 理查德$\cdot$西蒙  \\
	
	\midrule
	Richard Wagner   && \href{https://baike.baidu.com/item/%E7%90%86%E6%9F%A5%E5%BE%B7%C2%B7%E7%93%A6%E6%A0%BC%E7%BA%B3/2649053}{理查德$\cdot$瓦格纳}  \\
	
	\midrule
	right flexor (rF)   && 右屈肌  \\
	
	\midrule
	right flexor motor neurons (rFmn)   && 右屈肌运动神经元  \\
	
	\midrule
	right hemisphere (RH)  && 右半球  \\
	
	\midrule
	Rigid spine syndrome  && 脊柱强直综合症  \\
	
	\midrule
	Rio Hortega   && 瑞鸥$\cdot$霍特加  \\
	
	\midrule
	rippling muscle disease   && 波纹肌病  \\
	
	\midrule
	Rita Levi-Montalcini   && 丽塔$\cdot$列维-蒙塔尔奇尼  \\
	
	\midrule
	RNAse H   && 核糖核酸酶H  \\
	
	\midrule
	Roberta Klatzky   && 罗伯塔$\cdot$克莱兹基  \\
	
	\midrule
	Robert Burton   && 罗伯特$\cdot$波顿  \\
	
	\midrule
	Robert Edwards   && 罗伯特$\cdot$爱德华  \\
	
	\midrule
	Robert Lockhart   && 罗伯特$\cdot$洛哈特  \\
	
	\midrule
	Roberto Malinow   && 罗伯托$\cdot$马利诺  \\
	
	\midrule
	Robert Shprintzen   && 罗伯托$\cdot$什普林茨恩  \\
	
	\midrule
	Robert Stickgold   && 罗伯特$\cdot$史蒂克戈德  \\
	
	\midrule
	Rod MacKinnon   && 罗德$\cdot$麦金农  \\
	
	\midrule
	Rodolfo Llinás   && 鲁道夫$\cdot$利纳斯  \\
	
	\midrule
	robust nucleus of the archistriatum (RA)   && 古纹状体强健核  \\
	
	\midrule
	Rodolfo Llinas   && 鲁道夫$\cdot$伊利纳斯  \\
	
	\midrule
	rod bipolar (RB)  && 杆状双极细胞  \\
	
	\midrule
	rod cell   && 视杆细胞  \\
	
	\midrule
	Roger Albin   && 罗杰$\cdot$阿尔宾  \\
	
	\midrule
	Roland Johansson   && 罗兰$\cdot$约翰逊  \\
	
	\midrule
	Ron Harris-Warrick   && 罗恩$\cdot$哈里斯-瓦里克  \\
	
	\midrule
	Ron Rensink   && 罗恩$\cdot$伦辛克  \\
	
	\midrule
	René Spitz   && 雷诺$\cdot$史必兹  \\
	
	\midrule
	Roger Guillemin   && 罗杰$\cdot$吉莱明  \\
	
	\midrule
	Roger Sperry   && 罗杰$\cdot$斯佩里  \\
	
	\midrule
	Ronald Melzack   && \href{https://baike.baidu.com/item/%E6%A2%85%E5%B0%94%E6%89%8E%E5%85%8B/6474070?fr=ge_ala}{罗纳德$\cdot$梅尔扎克}  \\
	
	% 静纤毛小根
	\midrule
	Rootlet   && 小根  \\
	
	\midrule
	Ross Harrison   && 罗斯$\cdot$哈里逊  \\
	
	\midrule
	Rossetti   && 罗塞蒂  \\
	
	\midrule
	Rostral auditory cortex (R)   && 嘴侧听觉皮层  \\
	
	\midrule
	Rostrotemporal auditory cortex (R)   && 前颞听觉皮层 \\
	
	\midrule
	rough endoplasmic reticulum (RER)   && 糙面内质网 \\
	
	\midrule
	Rubinstein-Taybi syndrome  && \makecell[l]{阔拇指综合症, \\鲁宾斯坦-泰比综合症} \\
	
	\midrule
	Rubrospinal tract   && 红核脊髓束 \\
	
	\midrule
	Rudolph Leibel   && 鲁道夫$\cdot$利贝尔 \\
	
	\midrule
	Ruffini ending   && 鲁菲尼终末器 \\
	
	\midrule
	Ryanodine receptor   && 兰尼碱受体 \\
	
	\midrule
	saccadic   && 眼跳 \\
	
	\midrule
	saccharin preference (Sac)  && 糖精偏好 \\
	
	\midrule
	sagittal plane   && 矢状面 \\
	
	\midrule
	Sally-Anne test   && 萨莉-安妮测试 \\
	
	\midrule
	saltatory conduction   && 跳跃式传导 \\
	
	\midrule
	Sanford Palay   && 桑福德$\cdot$帕莱 \\
	
	\midrule
	Sanger Brown   && 桑各$\cdot$布朗 \\
	
	\midrule
	Santiago Ramony Cajal   && 圣地亚哥$\cdot$拉蒙-卡哈尔 \\
	
	\midrule
	Sarah Wilson   && 莎拉$\cdot$威尔逊 \\
	
	\midrule
	sarcomere length   && 肌节长度 \\
	
	\midrule
	Sarcoplasmic reticulum   && 肌质网 \\
	
	\midrule
	sartorius   && 缝匠肌 \\
	
	\midrule
	savant syndrome   && \href{https://baike.baidu.com/item/\%E5%AD%A6%E8%80%85%E7%BB%BC%E5%90%88%E7%97%87/4453123}{学者综合症} \\
	
	\midrule
	scala media   && 中阶 \\
	
	\midrule
	Scala tympani   && 鼓阶 \\
	
	\midrule
	Scala vestibuli   && 前庭阶 \\
	
	\midrule
	Schmidt   && 施密特 \\
	
	\midrule
	Secondary active transport   && 次级主动运输 \\
	
	\midrule
	secondary somatosensory cortex (S-II)   && 次级躯体感觉皮层 \\
	
	\midrule
	\makecell[l]{selective serotonin \\reuptake inhibitors (SSRI)}   && \makecell[l]{选择性5-羟色胺\\再吸收抑制剂} \\
	
	\midrule
	semaphorin   && 脑信号蛋白 \\
	
	\midrule
	sensor molecule   && 受体分子 \\
	
	\midrule
	sensorineural hearing loss   && 感音神经性聋 \\
	
	\midrule
	sensory threshold   && 受体分子 \\
	
	\midrule
	Serous gland   && 浆液腺	 \\
	
	\midrule
	Schaffer collateral pathway   && 谢弗侧支 \\
	
	\midrule
	Schenck   && 申克 \\
	
	\midrule
	Schwann cell   && 施旺细胞 \\
	
	\midrule
	Scott Sternson   && 斯科特$\cdot$斯特内森 \\
	
	\midrule
	Scoville   && 斯科维尔 \\
	
	\midrule
	Secondary generalization   && 次级泛化 \\
	
	% 脑电图上出现的一个或数个最明显的痫样放电部位,它可以是癫痫病理灶
	\midrule
	Seizure focus   && 致痫灶 \\
	
	\midrule
	Selenoprotein   && 硒蛋白 \\
	
	\midrule
	serine (Ser)   && 丝氨酸 \\
	
	\midrule
	serotonin reuptake transporter   && 血清素再摄取转运蛋白 \\
	
	\midrule
	serotonin transporter   && 血清素转运蛋白 \\
	
	\midrule
	set domain containing 1A (SETDIA)  && \makecell[l]{组蛋白H3赖氨酸4\\特异性甲基转移酶} \\
	
	\midrule
	sex-determining region on Y (SRY)   && Y染色体性别决定区 \\
	
	\midrule
	sex-linked inheritance   && \href{https://baike.baidu.com/item/\%E4%BC%B4%E6%80%A7%E9%81%97%E4%BC%A0/4078141}{伴性遗传} \\
	
	\midrule
	Sexual dimorphism   && 两性异形 \\
	
	\midrule
	\makecell[l]{sexually dimorphic nucleus of\\ the preoptic area (SDN-POA)}   && 视前区性二态核团 \\
	
	\midrule
	Seymour Benzer   && 西摩$\cdot$本泽 \\
	
	\midrule
	Seymour Kety   && 西摩$\cdot$凯帝 \\
	
	\midrule
	sharp-wave ripples   && 尖波涟漪 \\
	
	\midrule
	shell shock   && 战斗疲劳症 \\
	
	\midrule
	Shereshevski   && 舍雷舍夫斯 \\
	
	\midrule
	Short-term facilitation   && 短时程易化 \\
	
	\midrule
	Shosaku Numa   && 沼正作 \\
	
	\midrule
	sickle cell anemia   && \href{https://baike.baidu.com/item/%E9%95%B0%E5%88%80%E5%9E%8B%E7%BB%86%E8%83%9E%E8%B4%AB%E8%A1%80%E7%97%85}{镰状细胞贫血} \\
	
	\midrule
	Sid Kouider   && 西德$\cdot$欧伊德 \\
	
	\midrule
	Sidney Ochs   && 西德尼$\cdot$奥克斯 \\
	
	\midrule
	Short-term facilitation   && 短时程易化 \\
	
	\midrule
	Sigmund Freud   && 西格蒙德$\cdot$弗洛伊德 \\
	
	\midrule
	silent mutation   && \href{https://baike.baidu.com/item/%E6%B2%89%E9%BB%98%E7%AA%81%E5%8F%98/9716444}{沉默突变} \\
	
	% 寂静性伤害感受器广泛分布于皮肤、肌肉、关节和内脏中,通常的伤害性刺激并不能激活这些感受器,但炎症以及多种化学性刺激能使它们的发放阈值急剧降低,而可被较轻的伤害性刺激所激活,产生痛觉,在次级痛过敏和中枢敏感化中发挥重要作用。
	\midrule
	silent nociceptor  && 寂静性伤害性感受器 \\
	
	\midrule
	Silent synapse  && 静寂突触 \\
	
	\midrule
	\makecell[l]{single-photon emission \\computed tomography (SPECT)} && 单光子发射计算机断层扫描 \\
	
	\midrule
	sirtuins  && 乙酰化酶 \\
	
	\midrule
	sitter larvae  && 保姆幼虫 \\
	
	\midrule
	skelemins  && 骨架蛋白 \\
	
	\midrule
	Sliman Bensmaia  && 斯利曼$\cdot$本斯曼 \\
	
	\midrule
	Slowly adapting (SA) && 慢适应 \\
	
	\midrule
	slow-wave sleep (SWS) && 慢波睡眠 \\
	
	\midrule
	small noncoding RNA   && \href{https://wenku.baidu.com/view/60f60e595427a5e9856a561252d380eb63942371.html?_wkts_=1693876684239}{小非编码核糖核酸} \\
	
	\midrule
	small nuclear RNA (snRNA)   && \href{https://baike.baidu.com/item/%E5%B0%8F%E6%A0%B8RNA/10326792}{小核核糖核酸} \\
	
	\midrule
	Social anxiety disorder   && 社交焦虑症 \\
	
	\midrule
	Social Avoidance and Distress Scale   && 社交回避及苦恼量表 \\
	
	\midrule
	soleus muscle   && 比目鱼肌 \\
	
	\midrule
	Solitary nucleus   && 孤束核 \\
	
	\midrule
	\makecell[l]{soluble N-ethylmaleimide-sensitive \\factor attachment receptors (SNAREs)}   && \makecell[l]{可溶性N-乙基马来酰亚胺\\敏感因子附着受体} \\
	
	\midrule
	Somatosensory cortex   && 体感皮层 \\
	
	\midrule
	somatostatin (SS)  && 生长抑素 \\
	
	\midrule
	sonic hedgehog  && \href{https://baike.baidu.com/item/%E9%9F%B3%E7%8C%AC%E5%9B%A0%E5%AD%90/1561614}{音猬因子} \\
	
	\midrule
	Sonic Hedgehog Protein (Shh)  && 音速小子蛋白 \\
	
	\midrule
	sound-pressure level (SPL)  && \href{https://baike.baidu.com/item/%E5%A3%B0%E5%8E%8B%E7%BA%A7/2005936?fr=ge_ala}{声压级} \\
	
	\midrule
	spectrin-fodrin   && 血影蛋白-胞影蛋白 \\
	
	\midrule
	\makecell[l]{Speech Interpretation \& \\Recognition Interface (Siri)}  && 语音识别接口 \\
	
	\midrule
	spike discriminator   && 脉冲鉴别器 \\
	
	\midrule
	spike timing–dependent plasticity (STDP)  && 脉冲时序的可塑性 \\
	
	\midrule
	\makecell[l]{spinal nucleus of the\\ bulbocavernosus (SNB)}  && 球海绵体肌脊髓核 \\
	
	\midrule
	Spinal proprioceptor   && 脊髓本体感受器 \\
	
	\midrule
	Spinal trigeminal nucleus (STN)  && 三叉神经脊髓核 \\
	
	\midrule
	spine head   && 棘头 \\
	
	\midrule
	spinobulbar muscular atrophy (SBMA)   && 脊延髓肌萎缩症 \\
	
	% 3型
	\midrule
	spinocerebellar ataxias (SCAs)   && 脊髓小脑共济失调 \\
	
	\midrule
	Stanley Cohen   && 斯坦利$\cdot$科恩 \\
	
	\midrule
	Stanley Prusiner   && 史坦利$\cdot$布鲁希纳 \\
	
	\midrule
	Stanley S. Stevens   && 史坦利$\cdot$史蒂文斯 \\
	
	\midrule
	Stellate cell   && 星状细胞 \\
	
	\midrule
	Sten Grillner   && 斯滕$\cdot$格瑞那 \\
	
	\midrule
	Stephen Kuffler   && 斯蒂芬$\cdot$库夫勒 \\
	
	\midrule
	Stephen Liberles   && 斯蒂芬$\cdot$利伯莱斯 \\
	
	\midrule
	Steven Keele   && 史蒂文$\cdot$基尔 \\
	
	\midrule
	Steven Kuffler   && 史蒂文$\cdot$库夫勒 \\
	
	\midrule
	Steven McCarroll   && 史蒂文$\cdot$麦卡罗 \\
	
	\midrule
	Steven Waxman   && 史蒂文$\cdot$韦克斯曼 \\
	
	\midrule
	Steven Wise   && 史蒂文$\cdot$怀斯 \\
	
	\midrule
	stimulus (S)   && 刺激 \\
	
	\midrule
	stomatogastric ganglion  (STG)  && 口胃神经节 \\
	
	\midrule
	Stria terminalis  && 终纹 \\
	
	\midrule
	Strauss  && \href{https://baike.baidu.com/item/%E6%96%BD%E7%89%B9%E5%8A%B3%E6%96%AF%E5%AE%B6%E6%97%8F/9585595}{施特劳斯} \\
	
	\midrule
	Striatum (STR)  && 纹状体 \\
	
	\midrule
	Striola  && 弧形微纹 \\
	
	\midrule
	STS-temporalparietal junction   && 颞上沟-颞顶联合区 \\
	
	\midrule
	Subfornical organ   && 穹窿下器 \\
	
	\midrule
	sublenticular extended amygdala (SLEA)   && 近管状延伸杏仁核 \\
	
	\midrule
	subiculum   && 海马下托 \\
	
	\midrule
	Submodality   && 亚模态 \\
	
	\midrule
	subplate (SP)   && 底板 \\
	
	\midrule
	substance P   && P 物质(肽物质) \\
	
	\midrule
	\makecell[l]{substantia nigra and ventral tegmental \\area of the midbrain (SN/VTA)}   && 黑质/中脑腹侧被盖区 \\
	
	\midrule
	substantia nigra pars compacta (SNc)  && 黑质紧密区 \\
	
	\midrule
	substantia nigra pars reticulata (SNr)  && 黑质网状部 \\
	
	\midrule
	Substrate protein  && 底物蛋白 \\
	
	\midrule
	subthalamic nucleus (STN)   && 丘脑底核 \\
	
	\midrule
	subventricular zone (SVZ)   && 室下区 \\
	
	\midrule
	\makecell[l]{sudden unexpected death \\in epilepsy (SUDEP)}  && 癫痫突发意外死亡 \\
	
	\midrule
	sudden infant death syndrome (SIDS)  && 婴儿猝死综合症 \\
	
	\midrule
	Sulcus limitans   && 界沟 \\
	
	\midrule
	Samuel Detwiler   && 塞缪尔$\cdot$戴特威勒 \\
	
	\midrule
	Sylvian fissure   && 外侧裂 \\
	
	\midrule
	subfornical organ (SFO)  && 穹窿下器 \\
	
	\midrule
	superior cerebellar peduncle (SCP)  && 小脑上脚 \\
	
	\midrule
	Superior cervical ganglion   && 颈上神经节 \\
	
	\midrule
	superior colliculus (SC)  && 上丘 \\
	
	\midrule
	superior frontal language area (SFL) && 额上回语言区 \\
	
	\midrule
	superior longitudinal fasciculus   && 上纵束 \\
	
	\midrule
	superior oblique   && 上斜肌 \\
	
	\midrule
	Superior orbital fissure   && 眶上裂 \\
	
	\midrule
	superior rectus   && 上直肌 \\
	
	% 分隔了颞上回和颞中回
	\midrule
	superior temporal sulcus (STS)   && 颞上沟 \\
	
	\midrule
	superior view   && 俯视图 \\
	
	\midrule
	superficial radial nerves   && 浅桡神经 \\
	
	\midrule
	supplementary motor area (SMA)   && 辅助运动区 \\
	
	\midrule
	supplementary motor cortex (SMC)   && 辅助运动皮层 \\
	
	\midrule
	\makecell[l]{Suppressor of \\cytokine signaling 3 (SOCS3)}  && \makecell[l]{细胞因子信号通路\\抑制因子3} \\
	
	\midrule
	supraoptic nucleus (SON)  && 视上核 \\
	
	\midrule
	survival motor neuron (SMN) && 运动神经元生存 \\
	
	\midrule
	Susan Lederman && 苏珊$\cdot$莱德曼 \\
	
	\midrule
	Susumu Tonegawa && 利根川进 \\
	
	\midrule
	sympathetic nervous system (SNS) && 交感神经系统 \\
	
	\midrule
	synaptic boutons && 突触扣结 \\
	
	\midrule
	synaptic cleft && 突触间隙 \\
	
	\midrule
	\makecell[l]{Synaptobrevin (synaptic vesicle-associated\\ membrane protein, VAMP)}   && \makecell[l]{小突触囊泡蛋白\\ (突触小泡缔合性膜蛋白)} \\
	
	\midrule
	Synaptotagmin   && 突触结合蛋白 \\
	
	\midrule
	tabes dorsalis   && 	脊髓痨  \\
	
	\midrule
	Tadashi Isa   && 	伊佐正  \\
	
	\midrule
	Takao Hensch   && 	高雄$\cdot$亨施  \\
	
	\midrule
	Talairach space   && 	塔莱拉什空间  \\
	
	\midrule
	Talin   && 	踝蛋白  \\
	
	\midrule
	targetmembrane SNAREs t-SNAREs && 	\makecell{靶膜可溶性N-乙基\\马来酰亚胺敏感因子附着受体}  \\
	
	\midrule
	Taste bud   && 	味蕾  \\
	
	\midrule
	Taste pore   && 	味孔  \\
	
	\midrule
	Tegretol   && 	卡马西平  \\
	
	\midrule
	tetracycline transactivator (tTA)   && 四环素激活因子  \\
	
	\midrule
	tetracycline-responsive element (Teto, TRE)  && 四环素反应元件  \\
	
	\midrule
	tetraethylammonium (TEA)   && 四乙胺  \\
	
	\midrule
	Telethonin   && 视松蛋白  \\
	
	\midrule
	television camera (TV Camera)   && 电视摄像头  \\
	
	\midrule
	Temporal operculum   && 颞叶岛盖  \\
	
	\midrule
	template matching   && 模板匹配  \\
	
	\midrule
	temporal pole   && 颞极  \\
	
	\midrule
	Terje Lømo   && 泰耶$\cdot$洛莫  \\
	
	\midrule
	Terminal cisterna   && 终池  \\
	
	\midrule
	tetrabenazine  && 丁苯那嗪  \\
	
	\midrule
	tetramethylammonium (TMA)  && 四甲胺  \\
	
	\midrule
	tetrodotoxin (TTX)   && 河豚毒素  \\
	
	\midrule
	the biology of the mind   && 心理生物学  \\
	
	\midrule
	Theodor Schwann   && 西奥多$\cdot$施旺  \\
	
	\midrule
	theory of mind   && \href{https://baike.baidu.com/item/\%E5%BF%83%E6%99%BA%E7%90%86%E8%AE%BA/8719175}{心智理论}   \\
	
	\midrule
	Thermal nociceptor  && 温度性伤害感受器  \\
	
	\midrule
	Thomas Albright  && 托马斯$\cdot$奥尔布赖特  \\
	
	\midrule
	Thomas Bourgeron  && 托马斯$\cdot$波热龙  \\
	
	\midrule
	Thomas Elliott  && 托马斯$\cdot$艾略特  \\
	
	\midrule
	Thomas Graham Brown  && 托马斯$\cdot$格拉汉姆$\cdot$布朗  \\
	
	\midrule
	Thomas Hunt Morgan  && 托马斯$\cdot$亨特$\cdot$摩尔根  \\
	
	\midrule
	Thomas Reese  && 托马斯$\cdot$里斯  \\
	
	\midrule
	threonine-286 (Thr286) && 苏氨酸-286  \\
	
	\midrule
	Thunberg  && 桑伯格  \\
	
	\midrule
	Thymine (T)  && 胸腺嘧啶  \\
	
	\midrule
	Thyrotropin (TSH) && 促甲状腺激素  \\
	
	\midrule
	Thyrotropin-releasing hormone (TRH) && 促甲状腺素释放激素  \\
	
	% 胫(小腿)
	\midrule
	tibialis anterior (TIB) && 胫前肌  \\
	
	\midrule
	Tidal volume  && 潮气量  \\
	
	\midrule
	Tim Bliss  && 帝姆$\cdot$布利斯  \\
	
	\midrule
	tim gene  && 无节律基因  \\
	
	\midrule
	TIM  && 无节律蛋白  \\
	
	\midrule
	Timothy syndrome  && 蒂莫西综合症  \\
	
	\midrule
	Timothy Bliss  && 蒂莫西$\cdot$布利斯  \\
	
	\midrule
	Tip link  && 顶连  \\
	
	\midrule
	Thomas Elbert   && 托马斯$\cdot$艾尔伯特  \\
	
	\midrule
	Todd paralysisc   && 托德瘫痪  \\
	
	\midrule
	Todd Sacktor   && 托德$\cdot$萨克特  \\
	
	\midrule
	Tonegawa Susumu (Tonegawa)   && 利根川进  \\
	
	\midrule
	tonic activity   && 血管紧张性活动  \\
	
	\midrule
	tonotopic map   && 音调拓扑图  \\
	
	\midrule
	Torpedo marmorata   && 石纹电鳐  \\
	
	\midrule
	Torsten Wiesel   && 托斯坦$\cdot$威泽尔  \\
	
	\midrule
	Tourette syndrome   && 图雷特综合症  \\
	
	\midrule
	toxins   && 毒素  \\
	
	\midrule
	trace amine-associated receptors (TAAR)   && 微量胺相关受体  \\
	
	\midrule
	Transcranial magnetic stimulation (TMS)   && 经颅磁刺激  \\
	
	\midrule
	Transcription factors   && 转录因子  \\
	
	\midrule
	\makecell[l]{transcutaneous electrical \\nerve stimulation (TENS)}   && 经皮神经电刺激  \\
	
	\midrule
	transducin (T)   && 转导蛋白  \\
	
	\midrule
	transfer RNA (tRNA)   && \href{https://baike.baidu.com/item/\%E8%BD%AC%E8%BF%90RNA/5270033}{转运核糖核酸}  \\
	
	\midrule
	transforming growth factor (TGF$\beta$)  && 转化生长因子$\beta$  \\
	
	\midrule
	transient receptor potential (TRP)   && 瞬时受体电位  \\
	
	\midrule
	\makecell[l]{transient receptor potential \\ankyrin (TRPA)}   && 瞬时受体电位锚蛋白  \\
	
	\midrule
	\makecell[l]{transient receptor potential \\melastatin (TRPM)}   && 瞬时受体电位M型  \\
	
	\midrule
	\makecell[l]{transient receptor potential \\vanilloid (TRPV)}   && 瞬时受体电位香草醛受体  \\
	
	\midrule
	transit amplifying cell   && 过渡放大细胞  \\
	
	% 运用生物医药研究成果 (前临床及临床)以支持药物开发的过程
	\midrule
	Translational Science   && 转化科学  \\
	
	\midrule
	\makecell[l]{transmembrane AMPA receptor\\ regulatory proteins (TARP)}  && 跨膜AMPA受体调控蛋白  \\
	
	\midrule
	Transneuronal degeneration  && 跨神经元变性  \\
	
	\midrule
	transverse temporal gyri (Heschl's gyrus)   && 颞横回  \\
	
	\midrule
	transverse tubules   && 横小管  \\
	
	\midrule
	\makecell[l]{trans-(1S,3R)-1-amino\\-1,3-cyclopentanedicarboxylic \\acid (ACPD)}  && \makecell[l]{反-(1S,3R)-1-\\氨基-1,3-\\环戊烷二羧酸}  \\
	
	\midrule
	trapezoid body   && 斜方体  \\
	
	\midrule
	Trigeminal nerve   && 三叉神经  \\
	
	\midrule
	Trigger zone   && 触发区  \\
	
	\midrule
	Trochlear nerve (trochlear)   && 滑车神经  \\
	
	\midrule
	Troponin   && 肌钙蛋白  \\
	
	% 被模型预测为负的负样本 ;可以称作判断为假的正确率
	\midrule
	True negative (TN) && 真负  \\
	
	% 被模型预测为正的正样本;可以称作判断为真的正确率
	\midrule
	True positive (TP) && 真正  \\
	
	\midrule
	true-positive rate (TPR) && \href{https://baike.baidu.com/item/%E7%9C%9F%E9%98%B3%E6%80%A7%E7%8E%87/6345712}{真阳性率}  \\
	
	\midrule
	tryptophan (Trp)   && 色氨酸  \\
	
	\midrule
	tuberomammillary nucleus (TMN)  && 结节乳头核  \\
	
	\midrule
	tuberous sclerosis complex (TSC)  && 结节性硬化症  \\
	
	\midrule
	tubulin  && 微管蛋白  \\
	
	\midrule
	Tufted cell   && 簇状细胞  \\
	
	\midrule
	tumor necrosis factor (TNF)  && 肿瘤坏死因子  \\
	
	\midrule
	Turk-Browne  && 图尔克$\cdot$布朗  \\
	
	\midrule
	Two-pore-domain potassium channel (K2P)  && 双孔钾离子通道  \\
	
	\midrule
	\makecell[l]{type 5 metabotropic glutamate \\receptor (mGluR5)}  && 代谢5型谷氨酸受体  \\
	
	\midrule
	tyrosine kinases (trk)   && 酪氨酸激酶  \\
	
	\midrule
	Ubiquitin hydrolase  && 泛蛋白水解酶  \\
	
	% 医学上是个方位词。以手掌为例,靠小指一侧称为尺侧,靠拇指一侧称为桡侧。
	\midrule
	ulnar nerves  && 尺神经  \\
	
	\midrule
	ulnar-radial  && 尺侧-桡侧  \\
	
	\midrule
	unconditioned stimulus (US)  && 非条件刺激  \\
	
	\midrule
	uncoupling protein-1 (UCP1) && 解偶联蛋白1  \\
	
	\midrule
	Uridine (U)     &&  \href{https://baike.baidu.com/item/%E5%B0%BF%E8%8B%B7/4644045}{尿苷}  \\
	
	\midrule
	US Food and Drug Administration (FDA)     &&  美国食品和药物管理局  \\
	
	\midrule
	Usher syndrome     &&  遗传性耳聋-色素性视网膜炎综合症  \\
	
	\midrule
	utilization behavior   && 使用性行为  \\
	
	\midrule
	Uwe Frey   && 尤韦$\cdot$弗雷  \\
	
	\midrule
	V1   && 初级视觉皮层  \\
	
	\midrule
	vacuolar-type H$^+$-ATPase (V-ATPase)   && 空泡型氢离子三磷酸腺苷转运酶  \\
	
	\midrule
	vagus nerves   && 迷走神经  \\
	
	\midrule
	Valium   && 地西泮  \\
	
	\midrule
	\makecell[l]{vascular organ of the lamina \\terminalis (OVLT)}   && 终板血管器官  \\
	
	\midrule
	vasoactive intestinal peptide (VIP)  && 血管活性肠肽  \\
	
	\midrule
	vasopressin (VAS)  && 加压素  \\
	
	\midrule
	vasopressin receptors (V1a)  && 血管加压素受体  \\
	
	\midrule
	velocardiofacial syndrome (VCFS)   && 腭心面综合症  \\
	
	\midrule
	ventral caudate (vCD)   && 腹侧尾状核  \\
	
	\midrule
	\makecell[l]{ventral lateral aspect of the \\ventromedial hypothalamic nucleus (vlVMH)}   && 下丘脑腹内侧核的腹侧侧面  \\
	
	\midrule
	ventral intraparietal area (VIP)   && 顶内沟腹侧区  \\
	
	\midrule
	\makecell[l]{Ventral Nucleus of \\the Trapezoid Body(MNTB)}   && 斜方体腹侧核  \\
	
	\midrule
	ventral pallidum (VP)  && 腹侧苍白球  \\
	
	\midrule
	ventral posterior lateral (VPL)   && 腹后外侧  \\
	
	\midrule
	ventral posterior medial (VPM)   && 腹后内侧  \\
	
	\midrule
	Ventral posterior medial nucleus of thalamus   && 丘脑腹后核  \\
	
	\midrule
	ventral posterior superior (VPS)   && 腹后上  \\
	
	\midrule
	ventral premammillary nucleus (PMV)  && 腹侧乳头体核  \\
	
	\midrule
	ventral premotor cortex (PMv)   && 腹侧前运动皮层  \\
	
	\midrule
	ventral spinocerebellar tract (VSCT)   && 腹侧脊髓小脑束神经元  \\
	
	\midrule
	ventral subparaventricular zone (vSPZ)  && 腹侧脑室下区  \\
	
	% 中脑
	\midrule
	ventral tegmental area (VTA)   && 腹侧被盖区  \\
	
	\midrule
	ventricular zone (VZ)   && 脑室区  \\
	
	\midrule
	\makecell[l]{ventrolateral component of the \\ventromedial hypothalamus (VMHvl)}  && 下丘脑腹内侧的腹外侧部分  \\
	
	\midrule
	ventrolateral funiculus (VLF)   && 腹外侧索  \\
	
	\midrule
	ventrolateral prefrontal cortex (VLPFC, F47)   && 腹外侧前额叶皮层  \\
	
	\midrule
	\makecell[l]{ventrolateral preoptic nuclei\\ (VLPO, VLPFC, F47)}  && 腹外侧视前核  \\
	
	% 腹侧视丘
	\midrule
	Ventrolateral thalamus   && 丘脑腹外侧核  \\
	
	\midrule
	ventromedial hypothalamus (VMH)  && 腹内侧下丘脑  \\
	
	\midrule
	vergence movement   && 聚散运动  \\
	
	\midrule
	Vernier task   && 装游标的任务  \\
	
	\midrule
	Vernon Mountcastle   && 弗农$\cdot$芒卡斯尔  \\
	
	\midrule
	version movement   && 同向运动  \\
	
	\midrule
	vesicular ACh transporter (VAChT)   && 囊泡乙酰胆碱转运蛋白  \\
	
	\midrule
	\makecell[l]{vesicular glutamate \\transporter (V-GluT, VGlut)}   && 囊泡谷氨酸转运蛋白  \\
	
	\midrule
	vesicular glutamate transporter 2 (VGLUT2)   && 囊泡谷氨酸转运体2型  \\
	
	\midrule
	vesicular monoamine transporter (VMAT2)   && 囊泡单胺转运蛋白  \\
	
	\midrule
	\makecell[l]{vesicular soluble \\N-ethylmaleimide–sensitive \\factor attachment protein receptors\\ (v-SNAREs)}  && \makecell[l]{囊泡可溶性N-乙基马来酰\\亚胺敏感因子附着受体}  \\
	
	\midrule
	vestibular nuclei   && 前庭核  \\
	
	\midrule
	vestibulospinal tracts (VST)   && 前庭脊髓束  \\
	
	\midrule
	vestibulo-ocular reflexes (VOR)   && 前庭-眼动反射  \\
	
	\midrule
	Victor Horsley   && 维克多$\cdot$霍斯利  \\
	
	\midrule
	Victoria Abraira   && 维多利亚$\cdot$阿布雷拉  \\
	
	\midrule
	Viktor Gurfinkel   && 维克多$\cdot$古芬克尔  \\
	
	\midrule
	Viktor Hamburger   && 维克多$\cdot$汉堡  \\
	
	\midrule
	Vinculin   && 纽带蛋白  \\
	
	\midrule
	visual cue   && 视觉提示  \\
	
	\midrule
	visual discrimination   && 视觉辨别  \\
	
	\midrule
	visual posterior sylvian area (VPS)   && 视后外侧裂  \\
	
	\midrule
	visual receptive field (vRF)  && 视觉感受野  \\
	
	\midrule
	Visuotopic map  && 视觉拓扑映射  \\
	
	\midrule
	Voltage-clamp  && 电压钳  \\
	
	\midrule
	voltage-gated \ce{Ca^2+} channels (VGCCs) && 电压门控\ce{Ca^2+}通道  \\
	
	\midrule
	Vomeronasal organ (VNO)  && 犁鼻器  \\
	
	\midrule
	von Economo   && 冯$\cdot$伊克诺莫  \\
	
	\midrule
	Wade Marshall  && 韦德$\cdot$马歇尔  \\
	
	\midrule
	Wallerian degeneration  && 华勒氏变性  \\
	
	\midrule
	Wallerian degeneration slow (Wlds) && 华勒氏慢变性  \\
	
	\midrule
	Walter B. Cannon  && 沃尔特$\cdot$坎农  \\
	
	\midrule
	Walter Cannon  && 沃尔特$\cdot$坎农  \\
	
	\midrule
	Walter Gaskell  && 沃尔特$\cdot$盖斯凯尔  \\
	
	\midrule
	Walter Hess  && 沃尔特$\cdot$赫斯  \\
	
	\midrule
	Walter Nernst  && 沃尔特$\cdot$能斯特  \\
	
	\midrule
	Weber  && 韦伯  \\
	
	\midrule
	Weigert stain  && 威格特染色  \\
	
	\midrule
	Wernicke-Geschwind model  && 韦尼克$\cdot$格施温德模型  \\
	
	\midrule
	What pathway  && 内容通路  \\
	
	\midrule
	What/Who pathway/stream  && 内容通路  \\
	
	\midrule
	where pathway && 空间通路  \\
	
	\midrule
	Where/How pathway/stream && 空间通路  \\
	
	\midrule
	Wilder Penfield && 怀尔德$\cdot$潘菲尔德  \\
	
	% William的德语形式,男子名
	\midrule
	Wilhelm Erb && 威廉$\cdot$尔勃  \\
	
	% 德国柏林洪堡大学
	\midrule
	Wilhelm Sommer && 威廉$\cdot$索默  \\
	
	\midrule
	Wilhelm Wundt && 威廉$\cdot$冯特  \\
	
	\midrule
	William Bayliss && 威廉$\cdot$贝利斯  \\
	
	\midrule
	William de Kooning && 威廉$\cdot$德$\cdot$库宁  \\
	
	\midrule
	William James && \href{https://baike.baidu.com/item/%E5%A8%81%E5%BB%89%C2%B7%E8%A9%B9%E5%A7%86%E6%96%AF/6487016}{威廉$\cdot$詹姆斯}  \\
	
	\midrule
	William Newsome && 威廉$\cdot$纽瑟姆  \\
	
	\midrule
	Williams syndrome && 威廉综合症  \\
	
	\midrule
	William Willis && 威廉$\cdot$威利斯  \\
	
	\midrule
	Wilson Tanner && 威尔逊$\cdot$泰纳 \\
	
	\midrule
	Windsor side chair && 温莎侧椅 \\
	
	\midrule
	Winrich Freiwald && 温里奇$\cdot$弗赖瓦尔德 \\
	
	\midrule
	Irwin Feinberg && 欧文$\cdot$范伯格 \\
	
	% 沃尔夫管
	\midrule
	Wolffian duct && 中肾管  \\
	
	\midrule
	Wolfgang Köhler && \href{https://baike.baidu.com/item/%E6%B2%83%E5%B0%94%E5%A4%AB%E5%86%88%C2%B7%E6%9F%AF%E5%8B%92/6486904}{沃尔夫冈$\cdot$苛勒}  \\
	
	\midrule
	Wolfram Schultz && 沃尔夫勒姆$\cdot$舒尔茨  \\
	
	\midrule
	Woolsey && 伍尔西  \\
	
	\midrule
	Wylie Vale && 怀利$\cdot$瓦莱  \\
	
	\midrule
	X-linked recessive && \href{https://baike.baidu.com/item/X%E8%BF%9E%E9%94%81%E9%9A%90%E6%80%A7/53170799}{X连锁隐性}  \\
	
	\midrule
	Yasushi Miyashita && 宫下靖  \\
	
	\midrule
	Ying-hui Fu && 傅颖慧  \\
	
	\midrule
	Yngve Zotterman && 左特曼  \\
	
	\midrule
	Zarontin && 乙琥胺  \\
	
	\midrule
	Zinc finger 9 && 锌指 9  \\
	
	\midrule
	zona limitans intrathalamica (ZLI) && 限制性间脑区  \\
	
	\midrule
	hairy skin && 毛发皮肤 \\
	
	\midrule
	dorsal column nuclei  && 脊髓背根核
	\\
	
	\midrule
	subject  && 受试者
	\\
	
	\midrule
	papillary ridge  && 乳头脊
	\\
	
	\midrule
	proximal phalanges  && 近端指骨
	\\
	
	\midrule
	$\beta$-actin && 肌动蛋白  \\
	
	\midrule
	$\beta$-endorphin ($\beta$-END) && $\beta$-内啡肽  \\
	
	\midrule
	$\beta$-lipotropin ($\beta$-LPH) && $\beta$-促脂素  \\
	
	\midrule
	$\gamma$-secretase && $\gamma$分泌酶  \\
	
	
	
	\bottomrule  

\end{longtable}
%}
%\end{table}%

